%!TEX TS-program = xelatex
%!TEX encoding = UTF-8 Unicode
% Awesome CV LaTeX Template for CV/Resume
%
% This template has been downloaded from:
% https://github.com/posquit0/Awesome-CV
%
% Author:
% Claud D. Park <posquit0.bj@gmail.com>
% http://www.posquit0.com
%
% Template license:
% CC BY-SA 4.0 (https://creativecommons.org/licenses/by-sa/4.0/)
%


%-------------------------------------------------------------------------------
% CONFIGURATIONS
%-------------------------------------------------------------------------------
% A4 paper size by default, use 'letterpaper' for US letter
\documentclass[11pt, letterpaper]{awesome-cv}

% Configure page margins with geometry
\geometry{left=1.4cm, top=.8cm, right=1.4cm, bottom=1.8cm, footskip=.5cm}

% Color for highlights
% Awesome Colors: awesome-emerald, awesome-skyblue, awesome-red, awesome-pink, awesome-orange
%                 awesome-nephritis, awesome-concrete, awesome-darknight
\colorlet{awesome}{awesome-nephritis}
% Uncomment if you would like to specify your own color
% \definecolor{awesome}{HTML}{414141}

% Colors for text
% Uncomment if you would like to specify your own color
% \definecolor{darktext}{HTML}{414141}
% \definecolor{text}{HTML}{333333}
% \definecolor{graytext}{HTML}{5D5D5D}
% \definecolor{lighttext}{HTML}{999999}
% \definecolor{sectiondivider}{HTML}{5D5D5D}

% Set false if you don't want to highlight section with awesome color
\setbool{acvSectionColorHighlight}{false}

% Uncomment if you want to highlight with awesome color the whole section title
% \makeatletter
% \def\@sectioncolor{\color{awesome}}
% \makeatother

% If you would like to change the social information separator from a pipe (|) to something else
\renewcommand{\acvHeaderSocialSep}{\quad\textbar\quad}

%-------------------------------------------------------------------------------
%	PERSONAL INFORMATION
%	Comment any of the lines below if they are not required
%-------------------------------------------------------------------------------
% Available options: circle|rectangle,edge/noedge,left/right
% \photo[circle,edge,left]{/Users/matteo/matteorizzuto.github.io/images/matteo.jpeg}
\name{Matteo}{Rizzuto}
\position{Postdoctoral Fellow{\enskip\cdotp\enskip}Department of Biology, Concordia University\\
Visiting Scientist{\enskip\cdotp\enskip}Department of Biology, Memorial University of Newfoundland and Labrador}
% \address{}

\mobile{+1 (709) 986-5944}
\email{matteomrizzuto@gmail.com}
% \dateofbirth{December 12th, 1985}
\homepage{matteorizzuto.github.io}
% \github{matteorizzuto}
% \linkedin{matteorizzuto}
% \gitlab{gitlab-id}
% \stackoverflow{SO-id}{SO-name}
% \twitter{@MatteoRiz}
% \skype{matteolorin}
% \reddit{reddit-id}
% \medium{medium-id}
% \kaggle{kaggle-id}
% \googlescholar{googlescholar-id}{name-to-display}
%% \firstname and \lastname will be used
% \googlescholar{JkHYiIEAAAAJ}{Matteo Rizzuto}
\orcid{0000-0003-3065-9140}
% \researchgate{Matteo Rizzuto}
\extrainfo{\newline Pronouns: he/him/his}

% \quote{``Be the change that you want to see in the world."}


%-------------------------------------------------------------------------------
\begin{document}

% Print the header with above personal information
% Give optional argument to change alignment(C: center, L: left, R: right)
\makecvheader[L]


% Print the footer with 3 arguments(<left>, <center>, <right>)
% Leave any of these blank if they are not needed
\makecvfooter
  {\today}
  {Matteo Rizzuto~~~·~~~Publication list}
  {\thepage}


%-------------------------------------------------------------------------------
% SECTION TITLE
%-------------------------------------------------------------------------------
\cvsection{Publications}

%-------------------------------------------------------------------------------
% SUBSECTION TITLE
%-------------------------------------------------------------------------------
\descriptionstyle{{\textdagger} marks equal contributions. When listed as 2\textsuperscript{nd} or 3\textsuperscript{rd} author, I contributed to ideas, design, data collection and analysis, interpretation, and writing. When listed as 4\textsuperscript{th} author or later, I contributed to ideas, data collection, and writing.}

\begin{etaremune}[topsep=0pt,itemsep=1pt,partopsep=0pt,parsep=0pt]
  \renewcommand\labelenumi{\bfseries\theenumi .}
  \item \textcolor{awesome}{Rizzuto, M.}, Leroux, S. J., Schmitz, O. J. 2024. Rewiring the carbon cycle: a theoretical framework for animal-driven ecosystem carbon sequestration. \emph{Journal of Geophysical Research: Biogeosciences} \textbf{129}, e2024JG008026. \href{https://agupubs.onlinelibrary.wiley.com/doi/epdf/10.1029/2024JG008026?domain=author&token=JQVUWF9NSSQKUTSE44C6}{10.1029/2024JG008026}.\null\hfill\textbf{\textit{Research Spotlight}}
  \vspace{.3em}
  \begin{description}
    \item[\bodyfontlight Media coverage]
  \end{description}
  \begin{itemize}
    %\small
      \item Scharping, N. \href{https://doi.org/10.1029/2024EO240170}{Animals deserve to be included in global carbon cycle models, too}. \emph{Eos} \textbf{105}. 16 April 2024.
      \item Dinneen, J. \href{https://www.newscientist.com/article/2427674-animals-may-help-ecosystems-store-3-times-more-carbon-than-we-thought/}{Animals may help ecosystems store 3 times more carbon than we thought}. \emph{The New Scientist}. 19 April 2024.
  \end{itemize}
  \item \textcolor{awesome}{Rizzuto, M.}, Leroux, S. J., Schmitz, O. J., Vander Wal, E., Wiersma, Y. F., Heckford, T. R. 2024. Animal-vectored nutrient flows across resource gradients influence the nature of local and meta-ecosystem functioning. \emph{Ecological Modelling} \textbf{488}, 110570. doi: \href{https://doi.org/10.1016/j.ecolmodel.2023.110570}{10.1016/j.ecolmodel.2023.110570}.
  \item McLeod, A.M., Leroux, S.J., \textcolor{awesome}{Rizzuto, M.}, Leibold, M.A., Schiesari, L. 2024. Integrating ecosystem and contaminant models to predict the effects of ecosystem fluxes on contaminant dynamics. \emph{Ecosphere}, \textbf{15}(1): e4739. doi: \href{https://doi.org/10.1002/ecs2.4739}{10.1002/ecs2.4739}
  \item Heckford, T.R., Leroux, S.J., Vander Wal, E., \textcolor{awesome}{Rizzuto, M.}, Balluffi-Fry, J., Richmond, I.C., Wiersma, Y.F. 2022. Ecoregion and community structure influences on the foliar elemental niche of balsam fir (\textit{Abies balsamea} (L.) Mill.) and white birch (\textit{Betula papyrifera} Marshall). \emph{Ecology and Evolution} \textbf{12}, e9244. doi: \href{https://doi.org/10.1002/ece3.9244}{10.1002/ece3.9244}.
  \item Little, C.J.\textsuperscript{\textdagger}, \textcolor{awesome}{Rizzuto, M.}\textsuperscript{\textdagger}, Luhring, T.M., Monk, J.D., Nowicki, R.J., Paseka, R.E., Stegen, J.C., Symons, C.C., Taub, F.B., Yan, J.D.L. 2022. Movement with Meaning: Integrating Information into Meta-Ecology. \emph{Oikos} \textbf{8}, e08892.\\ doi: \href{https://doi.org/10.1111/oik.08892}{10.1111/oik.08892}. \null\hfill\textbf{\textit{Editor's Choice}}
  \item Balluffi-Fry, J., Leroux, S.J., Wiersma, Y.F., Richmond, I.C., Heckford, T.R., \textcolor{awesome}{Rizzuto, M.}, Kennah, J.L., Vander Wal, E. 2022. Integrating plant stoichiometry and feeding experiments: state-dependent forage choice and its implications on body mass. \emph{Oecologia} \textbf{198}(3), 579--591. doi: \href{https://rdcu.be/cAY5a}{10.1007/s00442-021-05069-5}.
  \item Richmond, I.C., Balluffi-Fry, J., Vander Wal, E., Leroux, S.J., \textcolor{awesome}{Rizzuto, M.}, Heckford, T.R., Kennah, J.L., Riefesel, G.R., Wiersma, Y.F. 2022. Individual snowshoe hares manage risk differently: integrating stoichiometric distribution models and foraging ecology. \emph{Journal of Mammalogy} \textbf{103}(1), 196--208. doi: \href{https://academic.oup.com/jmammal/advance-article/doi/10.1093/jmammal/gyab130/6441781?guestAccessKey=8f89e422-7fb9-4ce9-a9dc-ccf46f3dd0cc}{10.1093/jmammal/gyab130}.
  \item Heckford, T.R., Leroux, S.J., Vander Wal, E., \textcolor{awesome}{Rizzuto, M.}, Balluffi-Fry, J., Richmond, I.C., Wiersma, Y.F. 2022. Spatially explicit correlates of plant functional traits inform landscape patterns of resource quality. \emph{Landscape Ecology} \textbf{37}, 59--80. doi: \href{https://doi.org/10.1007/s10980-021-01334-3}{10.1007/s10980-021-01334-3}.
  \item \textcolor{awesome}{Rizzuto, M.}, Leroux, S.J., Vander Wal, E., Richmond, I.C., Heckford, T.R., Balluffi-Fry, J., Wiersma, Y.F. 2021. Forage stoichiometry predicts the home range size of a small terrestrial herbivore. \emph{Oecologia} \textbf{197}(2), 327--338.\\ doi: \href{https://rdcu.be/cSX31}{10.1007/s00442-021-04965-0}. \null\hfill\textbf{\textit{Winner, Hanski Prize 2021}}
  \item Ellis-Soto, D.\textsuperscript{\textdagger}, Ferraro, K.M.\textsuperscript{\textdagger}, \textcolor{awesome}{Rizzuto, M.}, Briggs, E., Monk, J.D., and Schmitz, O.J. 2021. A methodological roadmap to quantify animal-vectored spatial ecosystem subsidies. \emph{Journal of Animal Ecology} \textbf{90}(7), 1605--1622.\\ doi: \href{https://doi.org/10.1111/1365-2656.13538}{10.1111/1365-2656.13538}. \null\hfill\textbf{\textit{Winner, Sidnie Manton Award 2021}}
  \item Richmond, I.C., Leroux, S.J., Vander Wal, E., Heckford, T.R., \textcolor{awesome}{Rizzuto, M.}, Balluffi-Fry, J., Kennah, J., Wiersma, Y.F. 2021. Temporal variation and its drivers in the elemental traits of four boreal plant species. \emph{Journal of Plant Ecology} \textbf{14}(3), 398--413. doi: \href{https://doi.org/10.1093/jpe/rtaa103}{10.1093/jpe/rtaa103}.
  \item Balluffi-Fry, J., Leroux, S.J., Wiersma, Y.F., Heckford, T.R., \textcolor{awesome}{Rizzuto, M.}, Richmond, I.C., Vander Wal, E. 2020. Quantity-quality trade-offs revealed using a multiscale test of herbivore resource selection on elemental landscapes. \emph{Ecology and Evolution} \textbf{10}(24), 13847--13859. doi: \href{https://doi.org/10.1002/ece3.6975}{10.1002/ece3.6975}.
  \item \textcolor{awesome}{Rizzuto, M.}, Leroux, S.J., Vander Wal, E., Wiersma, Y.F., Heckford, T.R., Balluffi-Fry, J. 2019. Patterns and potential drivers of intraspecific variability in the body C, N, and P composition of a terrestrial vertebrate, the snowshoe hare (\textit{Lepus americanus}). \textit{Ecology and Evolution} \textbf{9}(24), 14453--14464. doi: \href{https://doi.org/10.1002/ece3.5880}{10.1002/ece3.5880}.
  \item \textcolor{awesome}{Rizzuto, M.}, Carbone, C., and Pawar, S. 2018. Foraging constraints reverse the scaling of activity time in carnivores. \emph{Nature Ecology and Evolution} \textbf{2}(2), 247--253. doi: \href{https://doi.org/10.1038/s41559-017-0386-1}{10.1038/s41559-017-0386-1}. \null\hfill\textbf{\textit{Cover~Story}}
  \vspace{.3em}
  \begin{description}
    \item[\bodyfontlight Media coverage]
  \end{description}
  \begin{itemize}
    %\small
      \item John, J. \href{https://wildlife.org/constantly-on-the-hunt-midsize-carnivores-face-unique-risks/}{Constantly on the hunt, midsize carnivores face unique risks}. \emph{The Wildlife Society}. 4 January 2018.
  \end{itemize}
\end{etaremune}


%-------------------------------------------------------------------------------
\end{document}
