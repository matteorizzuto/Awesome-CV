%-------------------------------------------------------------------------------
% SECTION TITLE
%-------------------------------------------------------------------------------
\cvsection{Publications}\\
\descriptionstyle{A {\textdagger} marks equal contributions. When listed as 2\textsuperscript{nd} or 3\textsuperscript{rd} author, I contributed to ideas, design, data collection and analysis, interpretation, and writing. When listed as 4\textsuperscript{th} author or later, I contributed to ideas, data collection, and writing.}

%-------------------------------------------------------------------------------
% CONTENT
%-------------------------------------------------------------------------------

\skilltypestyle{Peer reviewed} 

\begin{etaremune}[topsep=0pt,itemsep=1pt,partopsep=0pt,parsep=0pt]
  \renewcommand\labelenumi{\bfseries\theenumi .}
  \item \textcolor{awesome}{Rizzuto, M.}, Leroux, S. J., Schmitz, O. J. 2024. Rewiring the carbon cycle: a theoretical framework for animal-driven ecosystem carbon sequestration. \emph{Journal of Geophysical Research: Biogeosciences} \textbf{129}, e2024JG008026. \href{https://agupubs.onlinelibrary.wiley.com/doi/epdf/10.1029/2024JG008026?domain=author&token=JQVUWF9NSSQKUTSE44C6}{10.1029/2024JG008026}.\null\hfill\textbf{\textit{Research Spotlight}}
  \vspace{.3em}
  \begin{description}
    \item[\bodyfontlight Media coverage]
  \end{description}
  \begin{itemize}
    %\small
      \item Scharping, N. \href{https://doi.org/10.1029/2024EO240170}{Animals deserve to be included in global carbon cycle models, too}. \emph{Eos} \textbf{105}. 16 April 2024.
      \item Dinneen, J. \href{https://www.newscientist.com/article/2427674-animals-may-help-ecosystems-store-3-times-more-carbon-than-we-thought/}{Animals may help ecosystems store 3 times more carbon than we thought}. \emph{The New Scientist}. 19 April 2024.
  \end{itemize}
  \item \textcolor{awesome}{Rizzuto, M.}, Leroux, S. J., Schmitz, O. J., Vander Wal, E., Wiersma, Y. F., Heckford, T. R. 2024. Animal-vectored nutrient flows across resource gradients influence the nature of local and meta-ecosystem functioning. \emph{Ecological Modelling} \textbf{488}, 110570. doi: \href{https://doi.org/10.1016/j.ecolmodel.2023.110570}{10.1016/j.ecolmodel.2023.110570}.
  \item McLeod, A.M., Leroux, S.J., \textcolor{awesome}{Rizzuto, M.}, Leibold, M.A., Schiesari, L. 2024. Integrating ecosystem and contaminant models to predict the effects of ecosystem fluxes on contaminant dynamics. \emph{Ecosphere}, \textbf{15}(1): e4739. doi: \href{https://doi.org/10.1002/ecs2.4739}{10.1002/ecs2.4739}
  \item Heckford, T.R., Leroux, S.J., Vander Wal, E., \textcolor{awesome}{Rizzuto, M.}, Balluffi-Fry, J., Richmond, I.C., Wiersma, Y.F. 2022. Ecoregion and community structure influences on the foliar elemental niche of balsam fir (\textit{Abies balsamea} (L.) Mill.) and white birch (\textit{Betula papyrifera} Marshall). \emph{Ecology and Evolution} \textbf{12}, e9244. doi: \href{https://doi.org/10.1002/ece3.9244}{10.1002/ece3.9244}.
  \item Little, C.J.\textsuperscript{\textdagger}, \textcolor{awesome}{Rizzuto, M.}\textsuperscript{\textdagger}, Luhring, T.M., Monk, J.D., Nowicki, R.J., Paseka, R.E., Stegen, J.C., Symons, C.C., Taub, F.B., Yan, J.D.L. 2022. Movement with Meaning: Integrating Information into Meta-Ecology. \emph{Oikos} \textbf{8}, e08892.\\ doi: \href{https://doi.org/10.1111/oik.08892}{10.1111/oik.08892}. \null\hfill\textbf{\textit{Editor's Choice}}
  \item Balluffi-Fry, J., Leroux, S.J., Wiersma, Y.F., Richmond, I.C., Heckford, T.R., \textcolor{awesome}{Rizzuto, M.}, Kennah, J.L., Vander Wal, E. 2022. Integrating plant stoichiometry and feeding experiments: state-dependent forage choice and its implications on body mass. \emph{Oecologia} \textbf{198}(3), 579--591. doi: \href{https://rdcu.be/cAY5a}{10.1007/s00442-021-05069-5}.
  \item Richmond, I.C., Balluffi-Fry, J., Vander Wal, E., Leroux, S.J., \textcolor{awesome}{Rizzuto, M.}, Heckford, T.R., Kennah, J.L., Riefesel, G.R., Wiersma, Y.F. 2022. Individual snowshoe hares manage risk differently: integrating stoichiometric distribution models and foraging ecology. \emph{Journal of Mammalogy} \textbf{103}(1), 196--208. doi: \href{https://academic.oup.com/jmammal/advance-article/doi/10.1093/jmammal/gyab130/6441781?guestAccessKey=8f89e422-7fb9-4ce9-a9dc-ccf46f3dd0cc}{10.1093/jmammal/gyab130}.
  \item Heckford, T.R., Leroux, S.J., Vander Wal, E., \textcolor{awesome}{Rizzuto, M.}, Balluffi-Fry, J., Richmond, I.C., Wiersma, Y.F. 2022. Spatially explicit correlates of plant functional traits inform landscape patterns of resource quality. \emph{Landscape Ecology} \textbf{37}, 59--80. doi: \href{https://doi.org/10.1007/s10980-021-01334-3}{10.1007/s10980-021-01334-3}.
  \item \textcolor{awesome}{Rizzuto, M.}, Leroux, S.J., Vander Wal, E., Richmond, I.C., Heckford, T.R., Balluffi-Fry, J., Wiersma, Y.F. 2021. Forage stoichiometry predicts the home range size of a small terrestrial herbivore. \emph{Oecologia} \textbf{197}(2), 327--338.\\ doi: \href{https://rdcu.be/cSX31}{10.1007/s00442-021-04965-0}. \null\hfill\textbf{\textit{Winner, Hanski Prize 2021}}
  \item Ellis-Soto, D.\textsuperscript{\textdagger}, Ferraro, K.M.\textsuperscript{\textdagger}, \textcolor{awesome}{Rizzuto, M.}, Briggs, E., Monk, J.D., and Schmitz, O.J. 2021. A methodological roadmap to quantify animal-vectored spatial ecosystem subsidies. \emph{Journal of Animal Ecology} \textbf{90}(7), 1605--1622.\\ doi: \href{https://doi.org/10.1111/1365-2656.13538}{10.1111/1365-2656.13538}. \null\hfill\textbf{\textit{Winner, Sidnie Manton Award 2021}}
  \item Richmond, I.C., Leroux, S.J., Vander Wal, E., Heckford, T.R., \textcolor{awesome}{Rizzuto, M.}, Balluffi-Fry, J., Kennah, J., Wiersma, Y.F. 2021. Temporal variation and its drivers in the elemental traits of four boreal plant species. \emph{Journal of Plant Ecology} \textbf{14}(3), 398--413. doi: \href{https://doi.org/10.1093/jpe/rtaa103}{10.1093/jpe/rtaa103}.
  \item Balluffi-Fry, J., Leroux, S.J., Wiersma, Y.F., Heckford, T.R., \textcolor{awesome}{Rizzuto, M.}, Richmond, I.C., Vander Wal, E. 2020. Quantity-quality trade-offs revealed using a multiscale test of herbivore resource selection on elemental landscapes. \emph{Ecology and Evolution} \textbf{10}(24), 13847--13859. doi: \href{https://doi.org/10.1002/ece3.6975}{10.1002/ece3.6975}.
  \item \textcolor{awesome}{Rizzuto, M.}, Leroux, S.J., Vander Wal, E., Wiersma, Y.F., Heckford, T.R., Balluffi-Fry, J. 2019. Patterns and potential drivers of intraspecific variability in the body C, N, and P composition of a terrestrial vertebrate, the snowshoe hare (\textit{Lepus americanus}). \textit{Ecology and Evolution} \textbf{9}(24), 14453--14464. doi: \href{https://doi.org/10.1002/ece3.5880}{10.1002/ece3.5880}.
  \item \textcolor{awesome}{Rizzuto, M.}, Carbone, C., and Pawar, S. 2018. Foraging constraints reverse the scaling of activity time in carnivores. \emph{Nature Ecology and Evolution} \textbf{2}(2), 247--253. doi: \href{https://doi.org/10.1038/s41559-017-0386-1}{10.1038/s41559-017-0386-1}. \null\hfill\textbf{\textit{Cover~Story}}
  \vspace{.3em}
  \begin{description}
    \item[\bodyfontlight Media coverage]
  \end{description}
  \begin{itemize}
    %\small
      \item John, J. \href{https://wildlife.org/constantly-on-the-hunt-midsize-carnivores-face-unique-risks/}{Constantly on the hunt, midsize carnivores face unique risks}. \emph{The Wildlife Society}. 4 January 2018.
  \end{itemize}
\end{etaremune}

% \skilltypestyle{In progress}
% \setlist{nolistsep}
% \begin{itemize}[noitemsep]
  
% \end{itemize}

% \skilltypestyle{Reports}
% \setlist{nolistsep}
% \begin{itemize}[noitemsep] 
%   \item Schmitz, O. J., \textcolor{awesome}{Rizzuto, M.}, Retez, G. Animating the Carbon Cycle---Assessment of the feasibility of using European Bison to enhance ecosystem carbon capture and storage in the Carpathian Mountains of Romania. \emph{WWF Netherlands}, \emph{Global Rewilding Alliance}, \emph{Rewilding Europe}. 16 May 2024
%   \begin{description}
%     \item[\small \bodyfontlight Media coverage]
%   \end{description}
%   \begin{itemize}
%     \footnotesize
%       \item Green, G. \href{https://www.theguardian.com/environment/article/2024/may/15/bison-romania-tarcu-2m-cars-carbon-dioxide-emissions-aoe}{Herd of 170 bison could help store CO\textsubscript{2} equivalent of almost 2m cars, researchers say}. The Guardian, 15 May 2024. url: 
%   \end{itemize}

% \end{itemize}

\skilltypestyle{Outreach}
\setlist{nolistsep}
\begin{itemize}[noitemsep] 
  \item Wiersma, Y.F., Catto, N., Deal, C., Edinger, E., Evans, R., Geissinger, E., Hearn, C., Sun Lim, K., McCann, N., MacDonald, K., Meyer, A., Prosser, J., Quinn, D., Richmond, I.C., \textcolor{awesome}{Rizzuto, M.}, Roncal, J., Swain, M. 2020. The classroom goes virtual---experiences at Memorial University. \emph{Blog post}. url: \href{https://nllandscapeecology.com/blog-post-teaching-and-learning-remotely-time-to-read-4-min-45-s/}{https://nllandscapeecology.com}.
  \item Cagnacci, F., Rocca, M., Nicoloso, S., Ossi, F., Peters, W., Mancinelli, S., Valent, M., \textcolor{awesome}{Rizzuto, M.}, Hebblewhite, M. 2013. Il progetto 2C2T. \emph{Il Cacciatore Trentino}, 93, 4--15. url: \href{https://en.calameo.com/read/00214567355b1384f96d0}{https://en.calameo.com/}.
\end{itemize}


% \begin{refsection}
%   \nocite{Rizzuto2018}
  
%     \printbibliography[
%     heading=none
%     ]
% \end{refsection}

%-------------------------------------------------------------------------------
% SUBSECTION TITLE
%-------------------------------------------------------------------------------
% \cvsubsection{Conference Proceedings}

% \begin{refsection}
%     \nocite{Khan2014Prototyping}
  
%   \printbibliography[
%   heading=none, 
%   sorting=ydnt
%   ]
% \end{refsection}