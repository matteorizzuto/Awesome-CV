%!TEX TS-program = xelatex
%!TEX encoding = UTF-8 Unicode
% Awesome CV LaTeX Template for CV/Resume
%
% This template has been downloaded from:
% https://github.com/posquit0/Awesome-CV
%
% Author:
% Claud D. Park <posquit0.bj@gmail.com>
% http://www.posquit0.com
%
% Template license:
% CC BY-SA 4.0 (https://creativecommons.org/licenses/by-sa/4.0/)
%


%-------------------------------------------------------------------------------
% CONFIGURATIONS
%-------------------------------------------------------------------------------
% A4 paper size by default, use 'letterpaper' for US letter
\documentclass[11pt, letterpaper]{awesome-cv}

% Configure page margins with geometry
\geometry{left=1.4cm, top=.8cm, right=1.4cm, bottom=1.8cm, footskip=.5cm}

% Color for highlights
% Awesome Colors: awesome-emerald, awesome-skyblue, awesome-red, awesome-pink, awesome-orange
%                 awesome-nephritis, awesome-concrete, awesome-darknight
\colorlet{awesome}{awesome-nephritis}
% Uncomment if you would like to specify your own color
% \definecolor{awesome}{HTML}{414141}

% Colors for text
% Uncomment if you would like to specify your own color
% \definecolor{darktext}{HTML}{414141}
% \definecolor{text}{HTML}{333333}
% \definecolor{graytext}{HTML}{5D5D5D}
% \definecolor{lighttext}{HTML}{999999}
% \definecolor{sectiondivider}{HTML}{5D5D5D}

% Set false if you don't want to highlight section with awesome color
\setbool{acvSectionColorHighlight}{false}

% Uncomment if you want to highlight with awesome color the whole section title
% \makeatletter
% \def\@sectioncolor{\color{awesome}}
% \makeatother

% If you would like to change the social information separator from a pipe (|) to something else
\renewcommand{\acvHeaderSocialSep}{\quad\textbar\quad}

%-------------------------------------------------------------------------------
%	PERSONAL INFORMATION
%	Comment any of the lines below if they are not required
%-------------------------------------------------------------------------------
% Available options: circle|rectangle,edge/noedge,left/right
% \photo[circle,edge,right]{matteo.png}
\name{Matteo}{Rizzuto}
\position{Visiting Scientist{\enskip\cdotp\enskip}Department of Biology, Memorial University of Newfoundland and Labrador}
% \address{}

\mobile{+1 (709) 986-5944}
\email{matteomrizzuto@gmail.com}
% \dateofbirth{December 12th, 1985}
\homepage{matteorizzuto.github.io}
% \github{matteorizzuto}
% \linkedin{matteorizzuto}
% \gitlab{gitlab-id}
% \stackoverflow{SO-id}{SO-name}
% \twitter{@MatteoRiz}
% \skype{matteolorin}
% \reddit{reddit-id}
% \medium{medium-id}
% \kaggle{kaggle-id}
% \googlescholar{googlescholar-id}{name-to-display}
%% \firstname and \lastname will be used
\googlescholar{JkHYiIEAAAAJ}{Matteo Rizzuto}
\orcid{0000-0003-3065-9140}
% \researchgate{Matteo Rizzuto}
\extrainfo{Pronouns: he/him/his}

% \quote{``Be the change that you want to see in the world."}


%-------------------------------------------------------------------------------
\begin{document}

% Print the header with above personal information
% Give optional argument to change alignment(C: center, L: left, R: right)
\makecvheader[L]


% Print the footer with 3 arguments(<left>, <center>, <right>)
% Leave any of these blank if they are not needed
\makecvfooter
  {\today}
  {Matteo Rizzuto~~~·~~~Curriculum Vitae}
  {\thepage}


%-------------------------------------------------------------------------------
%	CV/RESUME CONTENT
%	Each section is imported separately, open each file in turn to modify content
%-------------------------------------------------------------------------------
%-------------------------------------------------------------------------------
%	SECTION TITLE
%-------------------------------------------------------------------------------
\cvsection{Highlights}


%-------------------------------------------------------------------------------
%	CONTENT
%-------------------------------------------------------------------------------

\begin{itemize}
  \item Published 13 peer-reviewed papers, of which 4 as first author and one as shared first author, with a combined 136 citations and an h-index of 7 (i10-index of 5).
  \item Winner of \textit{Oecologia}'s 2021 Hanski prize for best student-authored paper in animal ecology (as first author) and of the \textit{Journal of Animal Ecology}'s 2021 Sidnie Manton award for best early-career contribution (as co-author).
\end{itemize}
%-------------------------------------------------------------------------------
%	SECTION TITLE
%-------------------------------------------------------------------------------
\cvsection{Current position}


%-------------------------------------------------------------------------------
%	CONTENT
%-------------------------------------------------------------------------------
\begin{cventries}

%---------------------------------------------------------
  \cventry
    {Schmitz Lab, School of the Environment, Yale University} % Organization
    {Postdoctoral Associate} % Job title
    {Present} % Date(s)
    {New Haven, USA} % Location
    {
      \begin{cvitems} % Description(s) of tasks/responsibilities
        % \item {Developing new mathematical models to assess how animals influence carbon budgets and nutrient cycling in terrestrial ecosystems}
        \item {Building mathematical models to investigate and quantify animal influences on ecosystem carbon cycling, and inform and expand nature-based climate change solutions}
        % \item {Developing theoretical frameworks to support and expand nature-based climate change solutions, with a focus on trophic rewilding}
      \end{cvitems} 
    }
\end{cventries}

%-------------------------------------------------------------------------------
%	SECTION TITLE
%-------------------------------------------------------------------------------
\cvsection{Education}


%-------------------------------------------------------------------------------
%	CONTENT
%-------------------------------------------------------------------------------
\begin{cventries}

%---------------------------------------------------------
  \cventry
    {Ph.D. in Biology} % Degree
    {Memorial University of Newfoundland and Labrador} % Institution
    {St. John's, NL, Canada} % Location
    {Apr. 2016 - Nov. 2021} % Date(s)
    {
      \begin{cvitems} % Description(s) bullet points
        \item {Thesis title: From elements to landscapes: the role of terrestrial consumers in ecosystem functioning}
        \item {Supervisor: Dr.~Shawn~J.~Leroux}
      \end{cvitems}
    }

  \cventry
    {M.Res. in Ecology, Evolution, and Conservation} % Degree
    {Imperial College London} % Institution
    {London, United Kingdom} % Location
    {Sept. 2013 - Sept. 2014} % Date(s)
    {
      \begin{cvitems} % Description(s) bullet points
        \item {Graduated with Distinction}
      \end{cvitems}
    }
  
  \cventry
    {M.Sc. in Evolution of Animal and Human Behavior} % Degree
    {University of Turin} % Institution
    {Turin, Italy} % Location
    {Mar. 2009 - Dec. 2012} % Date(s)
    {
    %
    }
  
  \cventry
    {B.Sc. in Biology} % Degree
    {University of Turin} % Institution
    {Turin, Italy} % Location
    {Oct. 2004 - Mar. 2009} % Date(s)
    {
    %
    }
%---------------------------------------------------------
\end{cventries}

%-------------------------------------------------------------------------------
% SECTION TITLE
%-------------------------------------------------------------------------------
\cvsection{Publications}

%-------------------------------------------------------------------------------
% SUBSECTION TITLE
%-------------------------------------------------------------------------------
\descriptionstyle{%
\textbf{Peer reviewed.} \textsuperscript{\textdagger} stands for equal contribution.
}

\descriptionstyle{%
  \begin{etaremune}
    \renewcommand\labelenumi{\bfseries\theenumi .}
    \item Heckford, T.R., Leroux, S.J., Vander Wal, E., \textbf{Rizzuto, M.}, Balluffi-Fry, J., Richmond, I.C., Wiersma, Y.F. \textcolor{awesome-darknight}{Ecoregion and community structure influences on the foliar elemental niche of balsam fir (\textit{Abies balsamea} (L.) Mill.) and white birch (\textit{Betula papyrifera} Marshall)}. \emph{Ecology and Evolution} \textbf{2022}(12), e9244. DOI: \href{https://doi.org/10.1002/ece3.9244}{10.1002/ece3.9244}
    \item Little, C.J.\textsuperscript{\textdagger}, \textbf{Rizzuto, M.}\textsuperscript{\textdagger}, Luhring, T.M., Monk, J.D., Nowicki, R.J., Paseka, R.E., Stegen, J.C., Symons, C.C., Taub, F.B., Yan, J.D.L. \textcolor{awesome-darknight}{Movement with Meaning: Integrating Information into Meta-Ecology}. \emph{Oikos} \textbf{2022}(8), e08892. DOI: \href{https://doi.org/10.1111/oik.08892}{10.1111/oik.08892} \null\hfill\textbf{\textit{Editor's Choice}}
  \end{etaremune}
}


\begin{refsection}
  \nocite{Rizzuto2018}
  
    \printbibliography[
    heading=none
    ]
\end{refsection}

%-------------------------------------------------------------------------------
% SUBSECTION TITLE
%-------------------------------------------------------------------------------
% \cvsubsection{Conference Proceedings}

% \begin{refsection}
%     \nocite{Khan2014Prototyping}
  
%   \printbibliography[
%   heading=none, 
%   sorting=ydnt
%   ]
% \end{refsection}
%-------------------------------------------------------------------------------
%	SECTION TITLE
%-------------------------------------------------------------------------------
\cvsection{Presentations}


%-------------------------------------------------------------------------------
%	CONTENT
%-------------------------------------------------------------------------------
\skilltypestyle{Conference Talks}

\begin{itemize}
  \setlength\itemsep{.25em}
  \item \textcolor{awesome}{Rizzuto, M.}, Leroux, S.J., Schmitz, O.J. (2024, Aug. 6--11) \emph{Modeling the zoogeochemical effects of herbivores on ecosystem carbon cycles.} ``For All Ecologists'' Ecological Society of America Annual Meeting, Portland, OR, USA. \textbf{\emph{Invited talk}}  
  \item \textcolor{awesome}{Rizzuto, M.}, Leroux, S.J., Schmitz, O.J. (2023, May 25) \emph{Modeling the zoogeochemical effects of herbivores on ecosystem carbon cycles.} 6\textsuperscript{th} Yale Postdoc Association Annual Symposium, New Haven, CT, USA.
  \item \textcolor{awesome}{Rizzuto, M.}, Leroux, S.J., Schmitz, O.J., Vander Wal, E., Wiersma, Y.F., Heckford, T.R. (2023, May 2--6 ) \emph{Movers and shakers: Animal-vectored nutrient flows across resource gradients influence local and meta-ecosystem functioning.} ``Ecological Models for Tomorrow's Solutions'' The International Society for Ecological Modelling Global Conference 2023, Toronto, ON, Canada.
  \item \textcolor{awesome}{Rizzuto, M.}, Leroux, S.J., Schmitz, O.J., Vander Wal, E., Wiersma, Y.F., Heckford, T.R. (2021, Aug. 2--6) \emph{Going against the flow: non-diffusive organismal movement influences local and meta-ecosystem functioning.} ``Vital Connections in Ecology'' Ecological Society of America Virtual Annual Meeting, Long Beach, CA, USA.
  \item \textcolor{awesome}{Rizzuto, M.}, Leroux, S.J., Vander Wal, E., Wiersma, Y.F., Heckford, T.R., Balluffi-Fry, J. (2018, Jul. 18--21) \emph{Ontogeny and Ecological Stoichiometry of Snowshoe hares (Lepus americanus) in the Boreal Forests of Newfoundland.} Canadian Society for Ecology and Evolution Annual General Meeting, Guelph, ON, Canada.
  \item \textcolor{awesome}{Rizzuto, M.}, Carbone, C., and Pawar, S. (2016, Jul. 7--11) \emph{Bio-mechanical constraints on foraging reverse the scaling of activity rate among carnivores.} Canadian Society for Ecology and Evolution Annual General Meeting, St. John's, NL, Canada.
\end{itemize}

\skilltypestyle{Conference Posters}

\begin{itemize}
  \item \textcolor{awesome}{Rizzuto, M.}, Leroux, S.J., Vander Wal, E., Wiersma, Y.F., Heckford, T.R., Balluffi-Fry, J. (2018, Jul. 21--27) \emph{Beyond Diffusion: Animal-Mediated Nutrient Transport at Different Spatial Scales.} ``Unifying Ecology Across Scales'' Gordon Research Seminar and Conference, Biddeford, ME, USA.
\end{itemize}

\skilltypestyle{Workshops}

\begin{itemize}
  \item \textcolor{awesome}{Rizzuto, M.} (4 Nov. 2022) \emph{Grammar of Graphics: ggplot2.} Part of SPRY: A Learning Community for Quantitative Skill-Sharing\\ Yale University, School of the Environment.
  \item \textcolor{awesome}{Rizzuto, M.} (14 Oct. 2022) \emph{(R)markdown: a brief tour.} Part of SPRY: A Learning Community for Quantitative Skill-Sharing\\ Yale University, School of the Environment.
\end{itemize}

\skilltypestyle{Outreach}

\begin{itemize}
  \item \textcolor{awesome}{Rizzuto, M.} (24 Sept. 2024) \emph{Understanding the evidence. The latest science on trophic rewilding.} ``Are wild animals the unsung heroes of climate action?'' Panel, International Fund for Animal Welfare, New York Climate Week.
\end{itemize}
\begin{itemize}
  \item \textcolor{awesome}{Rizzuto, M.} (26 Apr. 2024) \emph{How animals shape our world: ecosystems, nutrient cycling, and carbon capture.} Wolfram Ecological Research Colloquium.
\end{itemize}

\skilltypestyle{Skype-a-Scientist initiative}

\begin{itemize}
  \item Louisville Zoo, Louisville, KY, USA. (30 Apr. 2025) 
  \item Tri-West Middle School, Lizton, IN, USA. (4 Dec. 2024)
  \item Edward Jones Scientist Expo Week. (16 Apr. 2024)
  \item Hope Township Elementary School, Hope, NJ, USA. (22 Mar. 2023)
\end{itemize}


%-------------------------------------------------------------------------------
%	SECTION TITLE
%-------------------------------------------------------------------------------
\cvsection{Teaching}


%-------------------------------------------------------------------------------
%	CONTENT
%-------------------------------------------------------------------------------
\begin{cventries}

%---------------------------------------------------------
  \cventry
    {Guest Lecturer} % Role
    {Yale University} % Title
    {2022--2024} % Date(s)
    {New Haven, USA} % Location
    {
      \begin{cvsubentries}
       \cvsubentry
        {\href{https://ypa.yale.edu/skills-and-career-development/modern-instructor}{The Modern Instructor}}
        {\footnotesize Lecture series}
        {Fall 2023}
        {
         \begin{cvitems}
          \item Design and delivered two lectures on ecosystem ecology and climate change aimed at a generalist audience
         \end{cvitems} 
        }
       \cvsubentry
        {Ecosystems and Landscapes}
        {\footnotesize Graduate course}
        {Fall 2023}
        {
         \begin{cvitems}
          \item Lead instructor on one of five course modules, focused on linking biodiversity to biogeochemistry
         \end{cvitems} 
        }
       \cvsubentry
         {Industrial Ecology}
         {\footnotesize Graduate course}
         {Fall 2022}
         {
         \begin{cvitems}
          \item Crafted and delivered a lecture introducing ideas and theories from metabolic ecology to urban ecology
         \end{cvitems} 
         }
      \end{cvsubentries}
    }
%---------------------------------------------------------
  \cventry
    {Teaching Assistant and Guest Lecturer} % Role
    {Memorial University of Newfoundland and Labrador} % Title
    {2017--2020} % Date(s)
    {St. John's, Canada} % Location
    {
      \begin{cvsubentries}
       \cvsubentry
         {Models in Biology}
         {\footnotesize Graduate and Undergraduate course}
         {Winter 2020}
         {
         \begin{cvitems}
          \item Designed and delivered lectures on classical models in Ecology, models of species interactions, and meta-ecology models
         \end{cvitems} 
         }
       \cvsubentry
         {Principles of Ecology}
         {\footnotesize Undergraduate course}
         {Fall 2018}
         {
         \begin{cvitems}
          \item Managed the online learning part of the course, provided administrative and academic support to students
         \end{cvitems} 
         }
       \cvsubentry
         {Graduate Core Seminar}
         {\footnotesize Graduate course}
         {Fall 2018}
         {
         \begin{cvitems}
          \item Developed and delivered a seminar on Student-Supervisor Communications
         \end{cvitems} 
         }  
       \cvsubentry
         {Principles of Biology}
         {\footnotesize Undergraduate course}
         {Winter 2017}
         {
         \begin{cvitems}
          \item Assisted during lab-based lectures, marked lab reports, midterms, and invigilated final exams
         \end{cvitems} 
         }
      \end{cvsubentries}
    }
%---------------------------------------------------------
 \cventry
    {Teaching Assistant} % Role
    {Imperial College London} % Title
    {2014--2015} % Date(s)
    {London, UK} % Location
    {
      \begin{cvsubentries}
       \cvsubentry
         {Ecology}
         {\footnotesize Undergraduate course}
         {Spring 2015}
         {
         \begin{cvitems}
          \item Demonstrator for the course's field trip, helped students design, collect data, and run analyses for their final projects
         \end{cvitems} 
         }
       \cvsubentry
         {Behavioral Ecology}
         {\footnotesize Undergraduate course}
         {Winter 2015}
         {
         \begin{cvitems}
          \item Assisted with lab-based lectures, from experiment setup to helping students with lab data analysis
         \end{cvitems} 
         }
       \cvsubentry
         {Introduction to Biological Statistics}
         {\footnotesize Undergraduate course}
         {Fall 2014}
         {
         \begin{cvitems}
          \item Helped students learn base and advanced R language, facilitated Q\&A sessions ahead of final exams
         \end{cvitems} 
         }
       \cvsubentry
        {Statistics}
        {\footnotesize Graduate course}
        {Fall 2014}
        {
          \begin{cvitems}
          \item Helped students in developing and coding statistical analyses in R
         \end{cvitems}
        }  
       \cvsubentry
        {Macroecology and Climate Change}
        {\footnotesize Graduate course}
        {Fall 2014}
        {
          \begin{cvitems}
          \item Assisted students using the species distribution software Maxent
         \end{cvitems}
        }   
      \end{cvsubentries}
    }
%---------------------------------------------------------
\end{cventries}


%-------------------------------------------------------------------------------
% CONTENT---SHORT FORM ?
%-------------------------------------------------------------------------------
% \begin{cventries}

% %---------------------------------------------------------
%   \cventryteaching
%     {Ecosystem and Landscapes} % title
%     {Guest lecturer} % role
%     {Graduate course} % level
%     {Fall 2023} % Date(s)
%     {Yale University} % Location
%     {} % description

%       % \begin{cvitems}
%       %   \item Lead instructor on one of five course modules, focused on linking biodiversity to biogeochemistry
%       % \end{cvitems} 
    

%   \cventryteaching
%     {Industrial Ecology} % title
%     {Guest lecturer} % role
%     {Graduate course} % level
%     {Fall 2022} % date(s)
%     {Yale University} % Location
%     {} % description

%          % {
%          % \begin{cvitems}
%          %  \item Crafted and delivered a lecture introducing ideas and theories from metabolic ecology to urban ecology
%          % \end{cvitems} 
%          % }

% %---------------------------------------------------------
%   \cventryteaching
%     {Models in Biology}
%     {Guest Lecturer} % Role
%     {Graduate course} % Title
%     {Winter 2020} % Date(s)
%     {Memorial University} % Location
%     {} 

% %       \begin{cvitems}
% %           \item Designed and delivered lectures on classical models in Ecology, models of species interactions, and meta-ecology models
% %          \end{cvitems}   
% %       \begin{cvsubentries}

% %---------------------------------------------------------

%    \cventryteaching
%       {Principles of Ecology}
%       {Teaching Assistant}  
%       {Undergraduate course}
%       {Fall 2018}
%       {Memorial University} % Location
%       {}

% %          \begin{cvitems}
% %           \item Managed the online learning part of the course, provided administrative and academic support to students
% %          \end{cvitems} 

% %---------------------------------------------------------

%   \cventryteaching
%      {Graduate Core Seminar}
%      {Guest lecturer}
%      {Graduate course}
%      {Fall 2018}
%      {Memorial University} % Location
%      {}

% %         \begin{cvitems}
% %          \item Developed and delivered a seminar on Student-Supervisor Communications
% %         \end{cvitems} 

% %---------------------------------------------------------
         
%   \cventryteaching
%      {Principles of Biology}
%      {Teaching Assistant}  
%      {Undergraduate course}
%      {Winter 2017}
%      {Memorial University} % Location
%      {}

% %        \begin{cvitems}
% %          \item Assisted during lab-based lectures, marked lab reports, midterms, and invigilated final exams
% %        \end{cvitems} 

% %---------------------------------------------------------

%  \cventryteaching
%     {Ecology}
%     {Demonstrator} % Role
%     {Undergraduate course} % Title
%     {Spring 2015} % Date(s)
%     {Imperial College London} % Location
%     {}

% %          \begin{cvitems}
% %           \item Demonstrator for the course's field trip, helped students design, collect data, and run analyses for their final projects
% %          \end{cvitems} 

% %---------------------------------------------------------

%   \cventryteaching
%      {Behavioral Ecology}
%      {Demonstrator} % Role
%      {Undergraduate course}
%      {Winter 2015}
%      {Imperial College London} % Location
%      {}

% %        \begin{cvitems}
% %         \item Assisted with lab-based lectures, from experiment setup to helping students with lab data analysis
% %        \end{cvitems} 

% %---------------------------------------------------------
      
%   \cventryteaching
%      {Introduction to Biological Statistics}
%      {Demonstrator} 
%      {Undergraduate course}
%      {Fall 2014}
%      {Imperial College London} % Location
%      {}

% %          \begin{cvitems}
% %           \item Helped students learn base and advanced R language, facilitated Q\&A sessions ahead of final exams
% %          \end{cvitems} 

% %---------------------------------------------------------

%    \cventryteaching
%       {Statistics}
%       {Demonstrator}
%       {Graduate course}
%       {Fall 2014}
%       {Imperial College London} % Location
%       {}

% %         \begin{cvitems}
% %           \item Helped students in developing and coding statistical analyses in R
% %         \end{cvitems}

% %---------------------------------------------------------
           
%    \cventryteaching
%       {Macroecology and Climate Change}
%       {Demonstrator}  
%       {Graduate course}
%       {Fall 2014}
%       {Imperial College London} % Location
%       {}

% %        \begin{cvitems}
% %          \item Assisted students using the species distribution software Maxent
% %        \end{cvitems}  

% %---------------------------------------------------------
% \end{cventries}

%-------------------------------------------------------------------------------
%	SECTION TITLE
%-------------------------------------------------------------------------------
\cvsection{Research Appointments}\\
\descriptionstyle{Note: 
\emph{Memorial University} stands for Memorial University of Newfoundland and Labrador.}


%-------------------------------------------------------------------------------
%	CONTENT
%-------------------------------------------------------------------------------
\begin{cventries}

%---------------------------------------------------------
\cventry
    {Department of Biology, Concordia University} % Organization
    {Postdoctoral Fellow} % Job title
    {2024--Present} % Date(s)
    {St. John's, NL, Canada} % Location
    {}


    % \begin{cvitems} % Description(s) of tasks/responsibilities
    %     \item {Developed mathematical models to investigate animal-mediated effects on ecosystem carbon budgets}
    %     \item {Worked with non-governmental organizations to inform rewilding efforts and expand nature-based climate change solutions}
    %   \end{cvitems}

%---------------------------------------------------------
\cventry
    {Department of Biology, Memorial University} % Organization
    {Visiting Scientist} % Job title
    {2024--Present} % Date(s)
    {St. John's, NL, Canada} % Location
    {}


    % \begin{cvitems} % Description(s) of tasks/responsibilities
    %     \item {Developed mathematical models to investigate animal-mediated effects on ecosystem carbon budgets}
    %     \item {Worked with non-governmental organizations to inform rewilding efforts and expand nature-based climate change solutions}
    %   \end{cvitems}
%---------------------------------------------------------
\cventry
    {Schmitz Lab, School of the Environment, Yale University} % Organization
    {Postdoctoral Associate} % Job title
    {2022--2024} % Date(s)
    {New Haven, CT, USA} % Location
    {}


    % \begin{cvitems} % Description(s) of tasks/responsibilities
    %     \item {Developed mathematical models to investigate animal-mediated effects on ecosystem carbon budgets}
    %     \item {Worked with non-governmental organizations to inform rewilding efforts and expand nature-based climate change solutions}
    %   \end{cvitems}
%---------------------------------------------------------
  \cventry
    {Schmitz Lab, School of the Environment, Yale University} % Organization
    {Visiting Assistant in Research} % Job title
    {Feb.--Apr. 2019} % Date(s)
    {New Haven, CT, USA} % Location
    {}

      % \begin{cvitems} % Description(s) of tasks/responsibilities
      %   \item {Developed a meta-ecosystem model capturing the effects of animal-mediated nutrient transport across ecosystem boundaries}
      % \end{cvitems}  
    
%---------------------------------------------------------
  \cventry
    {Pawar Lab, Department of Life Sciences, Imperial College London---Silwood Park} % Organization
    {Research Assistant} % Job title
    {Jan.--Mar. 2016} % Date(s)
    {Ascot, Berks., UK} % Location
    {}

      % \begin{cvitems} % Description(s) of tasks/responsibilities
      %   \item {Collected and managed data for a meta-analysis of published studies of plant thermal responses}
      % \end{cvitems}  
    
%---------------------------------------------------------
  \cventry
    {Tsaobis Baboon Project, Zoological Society of London} % Organization
    {Research Assistant} % Job title
    {Jun.--Oct. 2015} % Date(s)
    {London, UK} % Location
    {}

      %  \begin{cvitems} % Description(s) of tasks/responsibilities
      %   \item {Field-based placement to collect behavioral, dietary, and population data from wild \textit{Chacma} baboons in the Namib Desert}
      % \end{cvitems}  
    
%---------------------------------------------------------
  \cventry
    {Roe \& Red Deer in Trentino and Technology Project, Fondazione Edmund Mach} % Organization
    {Research Assistant} % Job title
    {Jun.--Sept. 2013} % Date(s)
    {Trento, TN, Italy} % Location
    {}

      % \begin{cvitems} % Description(s) of tasks/responsibilities
      %   \item {Field-based placement to collect data on roe deer movement data, plant phenology, and pellet decay rate in the Eastern Alps}
      % \end{cvitems}   
    
%---------------------------------------------------------
  \cventry
    {Piedmont Wolf Project, Maritime Alps Nature Park} % Organization
    {Graduate Intern} % Job title
    {Jun.--Dec. 2011} % Date(s)
    {Entracque, CN, Italy} % Location
    {}

      % \begin{cvitems} % Description(s) of tasks/responsibilities
      %   \item {Designed and led a mixed laboratory and field-based study of wolves-chamois predator-prey interactions in the Western Alps}
      % \end{cvitems} 
    
%---------------------------------------------------------
  \cventry
    {Ethology Lab, University of Turin} % Organization
    {Undergraduate Intern} % Job title
    {2008--2009} % Date(s)
    {Turin, TO, Italy} % Location
    {}

      %  \begin{cvitems} % Description(s) of tasks/responsibilities
      %   \item {Bio-acoustic study of an individual's contribution to their family group song among \textit{Indri} lemurs}
      % \end{cvitems} 
    
%---------------------------------------------------------
\end{cventries}

%-------------------------------------------------------------------------------
%	SECTION TITLE
%-------------------------------------------------------------------------------
\cvsection{Service}\\
\descriptionstyle{Note: 
\emph{Memorial University} stands for Memorial University of Newfoundland and Labrador.}

%-------------------------------------------------------------------------------
%	CONTENT
%-------------------------------------------------------------------------------



\skilltypestyle{Manuscript reviewer} %\entrypositionstyle{for \emph{Royal Society Open Sciences}, \emph{Biological Conservation}, \emph{Science of the Total Environment}, \emph{Ecology and Evolution}}

\begin{cvhonors}
%---------------------------------------------------------
  \mycvhonor
    {Ecology and Evolution} % Position
    {} % Committee
    % {} % Location
    {2021, 2024} % Date(s)
    % {}

  \mycvhonor
    {Royal Society Open Sciences} % Position
    {} % Committee
    % {} % Location
    {2024} % Date(s)
    % {}  

  \mycvhonor
    {Biological Conservation} % Position
    {} % Committee
    % {} % Location
    {2021, 2022} % Date(s)
    % {}  

  \mycvhonor
    {Science of the Total Environment} % Position
    {} % Committee
    % {} % Location
    {2019} % Date(s)
    % {}
\end{cvhonors}

\skilltypestyle{Conferences organized}

\begin{cventries}
%---------------------------------------------------------
  \cventry
    {Yale University} % Position
    {6\textsuperscript{th} and 7\textsuperscript{th} Annual Postdoc Symposium} % Committee
    {2023--2024} % Location
    {New Haven, CT, USA} % Date(s)
    {}
\end{cventries}

\skilltypestyle{Volunteer and community work}

\begin{cventries}

% %---------------------------------------------------------
%   \cventry
%     {Yale Postdoctoral Association} % Affiliation/role
%     {Advisory Board Member} % Organization/group
%     {Present} % Location
%     {Yale University} % Date(s)
%     {
%       \begin{cvitems} % Description(s) of experience/contributions/knowledge
%         \item {Member of the Yale Postdoctoral Association Advisory Board, participating in monthly meetings to steer the Association's activities}
%       \end{cvitems}
%     }

%---------------------------------------------------------
  \cventry
    {Yale University} % Affiliation/role
    {Yale Postdoctoral Association} % Organization/group
    {2022--Present} % Location
    {New Haven, CT, USA} % Date(s)
    {
      \begin{cvitems}
        \item Buddy Program Mentor, Present
        \item Symposium Committee Co-coordinator, 2023
        \item Symposium Committee Member, 2022 
      \end{cvitems}
    }

      % \begin{cvitems} % Description(s) of experience/contributions/knowledge
      %   \item {Co-coordinated the YPA Symposium Committee in organizing the 6\textsuperscript{th} and 7\textsuperscript{th} Annual Yale Postdoc Symposium}
      %   \item {Member of the Yale Postdoctoral Association Executive Board, participating in monthly meetings to steer the Association's activities}
      % \end{cvitems}
    
%---------------------------------------------------------
  % \cventry
  %   {Yale Postdoctoral Association} % Affiliation/role
  %   {Buddy Program Mentor} % Organization/group
  %   {Present} % Location
  %   {Yale University} % Date(s)
  %   {}

    %   \begin{cvitems} % Description(s) of experience/contributions/knowledge
    %     \item {Offered advice, help, and support to incoming international postdocs}
    %   \end{cvitems}
    
%---------------------------------------------------------
  % \cventry
  %   {Yale Postdoctoral Association} % Affiliation/role
  %   {Symposium Committee Member} % Organization/group
  %   {2022--2023} % Location
  %   {Yale University} % Date(s)
  %   {}

      % \begin{cvitems} % Description(s) of experience/contributions/knowledge
      %   \item {Leader of the Scheduling subcommittee, in charge of developing the Symposium's schedule, outreach and social activities, keynote talk, and panel discussions}
      % \end{cvitems}
    
%---------------------------------------------------------
  \cventry
    {Memorial University} % Affiliation/role
    {Biology Graduate Student Association} % Organization/group
    {2017--2020} % Location
    {St. John's, NL, Canada} % Date(s)
    {
      \begin{cvitems}
          \item Communications officer, 2019--2020
          \item Chairperson, 2018--2019
          \item Seminar Co-coordinator, 2017--2018
      \end{cvitems}
      }

      % \begin{cvitems} % Description(s) of experience/contributions/knowledge
      %   \item {Managed the weekly newsletter, website, event advertisement campaigns, and social media presence of the association}
      % \end{cvitems}
    
%---------------------------------------------------------
  % \cventry
  %   {Biology Graduate Student Association} % Affiliation/role
  %   {Chairperson} % Organization/group
  %   {2018--2019} % Location
  %   {Memorial University} % Date(s)
  %   {

  %   }

      % \begin{cvitems} % Description(s) of experience/contributions/knowledge
      %   \item {Coordinated the activities of the Executive Committee, focusing on fundraising, the welfare and mental health of graduate students, and community outreach}
      %   \item {Liaised with the Head of the Department of Biology to coordinate activities and events}
      % \end{cvitems}
    
%---------------------------------------------------------
  % \cventry
  %   {Biology Graduate Student Association} % Affiliation/role
  %   {Seminar Co-coordinator} % Organization/group
  %   {2017--2018} % Location
  %   {Memorial University} % Date(s)
  %   {}

      % \begin{cvitems} % Description(s) of experience/contributions/knowledge
      %   \item {One-third of the Seminar team, in charge of approaching, scheduling, and managing internal and visiting speakers for the weekly seminar series of the Biology Department}
      % \end{cvitems}
    
%---------------------------------------------------------
  \cventry
    {Associazione O.A.S.I.---Operazione Mato Grosso di Torino} % Affiliation/role
    {Educator} % Organization/group
    {2000--2013} % Location
    {Turin, Italy} % Date(s)
    {}

      % \begin{cvitems} % Description(s) of experience/contributions/knowledge
      %   \item {Designed, scheduled, and run educational activities and community outreach events with and for children, teenagers, and young adults in Turin, Italy, and at the Hospital S{\~a}o Juli{\~a}o in Campo Grande, Mato Grosso do Sul, Brazil}
      % \end{cvitems}
    
%---------------------------------------------------------
\cventry
    {XX Winter Olympic and IX Paralympic Games} % Affiliation/role
    {Team Leader} % Organization/group
    {2005--2006} % Location
    {Turin, Italy} % Date(s)
    {}

      % \begin{cvitems} % Description(s) of experience/contributions/knowledge
      %   \item {Led, scheduled, and supervised the daily activities of a ten-strong team of volunteers within the International Relations and Services Protocol Crew at the Ice Hockey, Curling, and Olympic Ceremonies venues}
      % \end{cvitems}
    
%---------------------------------------------------------
\end{cventries}

%-------------------------------------------------------------------------------
%	SECTION TITLE
%-------------------------------------------------------------------------------
\cvsection{Professional Development}\\
\descriptionstyle{Note: 
\emph{Memorial University} stands for Memorial University of Newfoundland and Labrador.} 


%-------------------------------------------------------------------------------
%	CONTENT
%-------------------------------------------------------------------------------

Courses

\begin{cventries}

%---------------------------------------------------------
  \cventry
    {MIT Teaching Systems Lab} % Affiliation/role
    {Becoming a More Equitable Educator: Mindsets and Practices} % Organization/group
    {Mar.--Jun. 2020} % Location
    {Attended online at \href{www.edx.org}{edX.org}} % Date(s)
    {
      % \begin{cvitems} % Description(s) of experience/contributions/knowledge
      %   \item {Led the Scheduling subcommittee, in charge of planning and organizing the annual YPA symposium}
      % \end{cvitems}
    }

%---------------------------------------------------------
  \cventry
    {Centre for Innovation in Teaching and Learning} % Affiliation/role
    {Teaching Skills Enhancement Program} % Organization/group
    {Sept. 2019--Aug. 2020} % Location
    {Memorial University} % Date(s)
    {
      % \begin{cvitems} % Description(s) of experience/contributions/knowledge
      %   \item {Managed the Association's weekly newsletter, website, event advertisement campaigns, and social media presence}
      % \end{cvitems}
    }

%---------------------------------------------------------
\end{cventries}

Workshops

\begin{cventries}

%---------------------------------------------------------
  \cventry
    {Canadian Institute for Ecology and Evolution \& NSERC-CREATE ``Living Data Project''} % Affiliation/role
    {Reproducible Research through Open Science} % Organization/group
    {11 Jun. 2020} % Location
    {Attended online at \href{https://osf.io/p7r5d/}{osf.io}} % Date(s)
    {
      % \begin{cvitems} % Description(s) of experience/contributions/knowledge
      %   \item {Led the Scheduling subcommittee, in charge of planning and organizing the annual YPA symposium}
      % \end{cvitems}
    }

%---------------------------------------------------------
  \cventry
    {American Society for Engineering Education \& NSF INCLUDES Aspire Alliance} % Affiliation/role
    {Teaching Inclusively \& Equitably Online} % Organization/group
    {21 May 2020} % Location
    {Attended online} % Date(s)
    {
      % \begin{cvitems} % Description(s) of experience/contributions/knowledge
      %   \item {Managed the Association's weekly newsletter, website, event advertisement campaigns, and social media presence}
      % \end{cvitems}
    }

%---------------------------------------------------------
  \cventry
    {Centre for Innovation in Teaching and Learning} % Affiliation/role
    {H5P Maker Session} % Organization/group
    {24 Oct. 2019} % Location
    {Memorial University} % Date(s)
    {
      % \begin{cvitems} % Description(s) of experience/contributions/knowledge
      %   \item {Managed the Association's weekly newsletter, website, event advertisement campaigns, and social media presence}
      % \end{cvitems}
    }

%---------------------------------------------------------
  \cventry
    {Centre for Innovation in Teaching and Learning} % Affiliation/role
    {Community of Inquiry Coffee Break: Open Access} % Organization/group
    {23 Oct. 2019} % Location
    {Memorial University} % Date(s)
    {
      % \begin{cvitems} % Description(s) of experience/contributions/knowledge
      %   \item {Managed the Association's weekly newsletter, website, event advertisement campaigns, and social media presence}
      % \end{cvitems}
    }

%---------------------------------------------------------
  \cventry
    {Centre for Innovation in Teaching and Learning} % Affiliation/role
    {Open Access and Scholarly Publishing} % Organization/group
    {22 Oct. 2019} % Location
    {Memorial University} % Date(s)
    {
      % \begin{cvitems} % Description(s) of experience/contributions/knowledge
      %   \item {Managed the Association's weekly newsletter, website, event advertisement campaigns, and social media presence}
      % \end{cvitems}
    }

%---------------------------------------------------------
  \cventry
    {Enhanced Development of the Graduate Experience} % Affiliation/role
    {Four Things to Consider for Graduate Student Teaching} % Organization/group
    {8 Nov. 2018} % Location
    {Memorial University} % Date(s)
    {
      % \begin{cvitems} % Description(s) of experience/contributions/knowledge
      %   \item {Managed the Association's weekly newsletter, website, event advertisement campaigns, and social media presence}
      % \end{cvitems}
    }

%---------------------------------------------------------
  \cventry
    {The Carpentries} % Affiliation/role
    {AARMS CRG Software Carpentry Workshop} % Organization/group
    {27 May 2017} % Location
    {Memorial University} % Date(s)
    {
      % \begin{cvitems} % Description(s) of experience/contributions/knowledge
      %   \item {Managed the Association's weekly newsletter, website, event advertisement campaigns, and social media presence}
      % \end{cvitems}
    }

%---------------------------------------------------------

\end{cventries}

%-------------------------------------------------------------------------------
%	SECTION TITLE
%-------------------------------------------------------------------------------
\cvsection{Honors \& Awards}\\
\descriptionstyle{Note: 
\emph{Memorial University} stands for Memorial University of Newfoundland and Labrador.}

%-------------------------------------------------------------------------------
%	CONTENT
%-------------------------------------------------------------------------------
\begin{cvhonors}

%---------------------------------------------------------
  \mycvhonor
    {Hanski Prize} % Award
    {Oecologia} % Event
    % {} % Location
    {2021} % Date(s)
    % {} % Description
    % Awarded by the journal's Editorial Board to the best student-authored paper in animal ecology
%---------------------------------------------------------
  \mycvhonor
    {Mitacs Research Training Award} % Award
    {Mitacs, St. John's, Canada} % Event
    % {St. John's, NL, Canada} % Location
    {2020} % Date(s)
    % {} % Description
    % Awarded to graduate students to conduct research during the COVID-19 pandemic
%---------------------------------------------------------
  \mycvhonor
    {First Place--H5P Maker Session} % Award
    {Memorial University, St.~John's, Canada} % Event
    % {St. John's, NL, Canada} % Location
    {2019} % Date(s)
    % {} % Description
    % Awarded by the Centre for Innovation in Teaching \& Learning to the creator of the best HTML5 interactive learning object
%---------------------------------------------------------
  \mycvhonor
    {Mitacs Globalink Research Award} % Award
    {Mitacs, St. John's, Canada} % Event
    % {St. John's, NL, Canada} % Location
    {2019} % Date(s)
    % {} % Description
    % Awarded to senior graduate students to conduct research  abroad
%---------------------------------------------------------
  \mycvhonor
    {Dean's Doctoral Award} % Award
    {Memorial University, St. John's, Canada} % Event
    % {St. John's, NL, Canada} % Location
    {2016--2020} % Date(s)
    % {} % Description
    % Awarded to Ph.D. students based on academic excellence and research potential
%---------------------------------------------------------
  \mycvhonor
    {Graduated with Distinction} % Award
    {Imperial College London, London, UK} % Event
    % {London, United Kingdom} % Location
    {2014} % Date(s)
    % {} % Description
    % Awarded to the top 5\% of all Master's students in a cohort
%---------------------------------------------------------
  \mycvhonor
    {Erasmus-LLP Scholarship} % Award
    {University of Turin, Turin, Italy} % Event
    % {Turin, Italy} % Location
    {2010} % Date(s)
    % {} % Description
    % Awarded to students to take part in the Erasmus exchange program of the European Union
%--------------------------------------------------------- 
\end{cvhonors}
% %-------------------------------------------------------------------------------
%	SECTION TITLE
%-------------------------------------------------------------------------------
\cvsection{Extracurricular Activity}\\
\descriptionstyle{Note: 
\emph{Memorial University} stands for Memorial University of Newfoundland and Labrador.}

%-------------------------------------------------------------------------------
%	CONTENT
%-------------------------------------------------------------------------------
\begin{cventries}

%---------------------------------------------------------
  \cventry
    {Yale Postdoctoral Association} % Affiliation/role
    {Symposium Committee Member} % Organization/group
    {Present} % Location
    {Yale University} % Date(s)
    {
      \begin{cvitems} % Description(s) of experience/contributions/knowledge
        \item {Led the Scheduling subcommittee, in charge of planning and organizing the annual YPA symposium}
      \end{cvitems}
    }

%---------------------------------------------------------
  \cventry
    {Biology Graduate Student Association} % Affiliation/role
    {Communications Officer} % Organization/group
    {2019--2020} % Location
    {Memorial University} % Date(s)
    {
      \begin{cvitems} % Description(s) of experience/contributions/knowledge
        \item {Managed the Association's weekly newsletter, website, event advertisement campaigns, and social media presence}
      \end{cvitems}
    }

%---------------------------------------------------------
  \cventry
    {Biology Graduate Student Association} % Affiliation/role
    {Chairperson} % Organization/group
    {2018--2019} % Location
    {Memorial University} % Date(s)
    {
      \begin{cvitems} % Description(s) of experience/contributions/knowledge
        \item {Coordinated the activities of the Executive Committee, focusing on fundraising, graduate students' welfare and mental health, and community outreach}
        \item {Liaised with the Head of the Department of Biology to coordinate activities and events}
      \end{cvitems}
    }

%---------------------------------------------------------
  \cventry
    {Biology Graduate Student Association} % Affiliation/role
    {Seminar Coordinator} % Organization/group
    {2017--2018} % Location
    {Memorial University} % Date(s)
    {
      \begin{cvitems} % Description(s) of experience/contributions/knowledge
        \item {One-third of the Seminar team, in charge of approaching, scheduling, and managing internal and external speakers for the weekly seminar series of the Biology Department}
      \end{cvitems}
    }

%---------------------------------------------------------
  \cventry
    {Associazione O.A.S.I.---Operazione Mato Grosso di Torino} % Affiliation/role
    {Youth Educator} % Organization/group
    {2000--2013} % Location
    {Turin, Italy} % Date(s)
    {
      \begin{cvitems} % Description(s) of experience/contributions/knowledge
        \item {Designed, scheduled, and run educational activities and community outreach events with and for children, teenagers, and young adults in Turin, Italy, and at the Hospital S{\~a}o Juli{\~a}o in Campo Grande, Mato Grosso do Sul, Brazil}
      \end{cvitems}
    }

%---------------------------------------------------------
\cventry
    {XX Winter Olympic and IX Paralympic Games} % Affiliation/role
    {Team Leader} % Organization/group
    {2005--2006} % Location
    {Turin, Italy} % Date(s)
    {
      \begin{cvitems} % Description(s) of experience/contributions/knowledge
        \item {Led, scheduled, and supervised the daily activities of a ten-strong team of volunteers within the International Relations and Services Protocol Crew at the Ice Hockey, Curling, and Ceremonies Olympic venues}
      \end{cvitems}
    }

%---------------------------------------------------------
\end{cventries}

%-------------------------------------------------------------------------------
%	SECTION TITLE
%-------------------------------------------------------------------------------
\cvsection{Professional Affiliations}


%-------------------------------------------------------------------------------
%	CONTENT
%-------------------------------------------------------------------------------
\begin{cvhonors}

%---------------------------------------------------------
  \mycvhonor
    {International Society for Ecological Modelling} % Committee
    {} % Position
    % {}
    {2023--2025} % Location
    % {} % Date(s)
    
%---------------------------------------------------------
  \mycvhonor
    {Ecological Society of America} % Position
    {} % Committee
    % {} % Location
    {2021--2023} % Date(s)
    % {}

%---------------------------------------------------------
  \mycvhonor
    {Canadian Society for Ecology and Evolution} % Position
    {} % Committee
    % {} % Location
    {2016--2021} % Date(s)
    % {}

%---------------------------------------------------------
\end{cvhonors}

% %-------------------------------------------------------------------------------
%	SECTION TITLE
%-------------------------------------------------------------------------------
\cvsection{Conferences Attended}


%-------------------------------------------------------------------------------
%	CONTENT
%-------------------------------------------------------------------------------
\begin{cventries}

%---------------------------------------------------------
  \cventry
    {Ecological Society of America, Virtual Annual Meeting} % Position
    {Vital Connections in Ecology} % Committee
    {2--6 Aug. 2021} % Location
    {Long Beach, CA, USA} % Date(s)
    {
      %
    }

%---------------------------------------------------------
  \cventry
    {Gordon Research Seminar and Conference} % Position
    {Unifying Ecology Across Scales} % Committee
    {21--27 Jul. 2018} % Location
    {Biddeford, ME, USA} % Date(s)
    {
      %
    }

%---------------------------------------------------------
  \cventry
    {Annual General Meeting} % Position
    {Canadian Society for Ecology and Evolution} % Committee
    {18--21 Jul. 2018} % Location
    {Guelph, ON, Canada} % Date(s)
    {
      %
    }

%---------------------------------------------------------
  \cventry
    {Canadian Society for Ecology and Evolution} % Position
    {Annual General Meeting} % Committee
    {7--11 Jul. 2016} % Location
    {St. John's, NL, Canada} % Date(s)
    {
      %
    }

%---------------------------------------------------------
  \cventry
    {Zoological Society of London} % Position
    {From Energetics to Macro Ecology: Carnivore Responses to Environmental Change} % Committee
    {14--15 Nov. 2013} % Location
    {London, UK} % Date(s)
    {
      %
    }

%---------------------------------------------------------
  \cventry
    {Euroscience} % Position
    {ESOF---Euroscience Open Forum} % Committee
    {2--7 Jul. 2010} % Location
    {Turin, Italy} % Date(s)
    {
      %
    }

%---------------------------------------------------------
  \cventry
    {Italian Primatological Society} % Position
    {XIX Congress} % Committee
    {1--3 Apr. 2009} % Location
    {Asti, Italy} % Date(s)
    {
      %
    }

%---------------------------------------------------------
\end{cventries}

% %-------------------------------------------------------------------------------
%	SECTION TITLE
%-------------------------------------------------------------------------------
\cvsection{Skills}


%-------------------------------------------------------------------------------
%	CONTENT
%-------------------------------------------------------------------------------
\begin{cvskills}

%---------------------------------------------------------
  \cvskill
    {Programming} % Category
    {R, Wolfram Mathematica, LaTeX, Bash, HTML} % Skills

%---------------------------------------------------------
  \cvskill
    {Software} % Category
    {Version Control (Git, GitHub), RStudio, Atom, Sublime Text, QGIS, Inkscape} % Skills

%---------------------------------------------------------
  \cvskill
    {Platforms} % Category
    {macOS, Windows} % Skills

%---------------------------------------------------------
  \cvskill
    {Languages} % Category
    {Italian (Native), English (Fluent), French (Basic), Spanish (Basic), Polish (Basic)} % Skills

%---------------------------------------------------------
\end{cvskills}

% %-------------------------------------------------------------------------------
%	SECTION TITLE
%-------------------------------------------------------------------------------
\cvsection{Certificates}


%-------------------------------------------------------------------------------
%	CONTENT
%-------------------------------------------------------------------------------
\begin{cventries}

%---------------------------------------------------------
  \cventry
    {Canadian Red Cross} % Position
    {Wilderness and Remote First Aid} % Committee
    {2016--2019} % Location
    {St. John's, NL, Canada} % Date(s)
    {
      %
    }

%---------------------------------------------------------
  \cventry
    {Memorial University of Newfoundland and Labrador} % Position
    {WHMIS and Lab Safety} % Committee
    {2017} % Location
    {St. John's, NL, Canada} % Date(s)
    {
      %
    }

%---------------------------------------------------------
  \cventry
    {Marlin Training Ltd.} % Position
    {Basic Outdoor First Aid} % Committee
    {2014} % Location
    {London, United Kingdom} % Date(s)
    {
      %
    }

%---------------------------------------------------------
  \cventry
    {British Council} % Position
    {International English Language Testing System} % Committee
    {2013} % Location
    {Turin, Italy} % Date(s)
    {
      Grade: 8.5
    }

%---------------------------------------------------------
  \cventry
    {University of Turin} % Position
    {International Computer Driving Licence} % Committee
    {2007} % Location
    {Turin, Italy} % Date(s)
    {
      %
    }

%---------------------------------------------------------
\end{cventries}

% %-------------------------------------------------------------------------------
%	SECTION TITLE
%-------------------------------------------------------------------------------
\cvsection{Interests}


%-------------------------------------------------------------------------------
%	CONTENT
%-------------------------------------------------------------------------------

%---------------------------------------------------------
\begin{cvskills}
  \skillsetstyle{Hiking, Photography, Yoga, Watercolor painting}
\end{cvskills}
% \newpage
% %-------------------------------------------------------------------------------
%	SECTION TITLE
%-------------------------------------------------------------------------------
\cvsection{References}


%-------------------------------------------------------------------------------
%	CONTENT
%-------------------------------------------------------------------------------
\cvdoublecolumn
{\cvreference{Dr. Oswald J. Schmitz}
  {School of the Environment}
  {Yale University}
  {New Haven, CT, USA}
  {oswald.schmitz@yale.edu}
}
{\cvreference{Dr. Shawn J. Leroux}
  {Department of Biology}
  {Memorial University of Newfoundland}
  {St. John's, NL, Canada}
  {sleroux@mun.ca}
}
{\cvreference{Dr. Yolanda F. Wiersma}
  {Department of Biology}
  {Memorial University of Newfoundland}
  {St. John's, NL, Canada}
  {ywiersma@mun.ca}
}
{\cvreference{Dr.~Samraat Pawar}
  {Department of Life Sciences}
  {Imperial College London---Silwood Park}
  {Ascot, Berkshire, UK}
  {s.pawar@imperial.ac.uk}
}


%-------------------------------------------------------------------------------
\end{document}
