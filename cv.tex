%!TEX TS-program = xelatex
%!TEX encoding = UTF-8 Unicode
% Awesome CV LaTeX Template for CV/Resume
%
% This template has been downloaded from:
% https://github.com/posquit0/Awesome-CV
%
% Author:
% Claud D. Park <posquit0.bj@gmail.com>
% http://www.posquit0.com
%
% Template license:
% CC BY-SA 4.0 (https://creativecommons.org/licenses/by-sa/4.0/)
%


%-------------------------------------------------------------------------------
% CONFIGURATIONS
%-------------------------------------------------------------------------------
% A4 paper size by default, use 'letterpaper' for US letter
\documentclass[11pt, letterpaper]{awesome-cv}

% Configure page margins with geometry
\geometry{left=1.4cm, top=.8cm, right=1.4cm, bottom=1.8cm, footskip=.5cm}

% Color for highlights
% Awesome Colors: awesome-emerald, awesome-skyblue, awesome-red, awesome-pink, awesome-orange
%                 awesome-nephritis, awesome-concrete, awesome-darknight
\colorlet{awesome}{awesome-nephritis}
% Uncomment if you would like to specify your own color
% \definecolor{awesome}{HTML}{414141}

% Colors for text
% Uncomment if you would like to specify your own color
% \definecolor{darktext}{HTML}{414141}
% \definecolor{text}{HTML}{333333}
% \definecolor{graytext}{HTML}{5D5D5D}
% \definecolor{lighttext}{HTML}{999999}
% \definecolor{sectiondivider}{HTML}{5D5D5D}

% Set false if you don't want to highlight section with awesome color
\setbool{acvSectionColorHighlight}{false}

% Uncomment if you want to highlight with awesome color the whole section title
% \makeatletter
% \def\@sectioncolor{\color{awesome}}
% \makeatother

% If you would like to change the social information separator from a pipe (|) to something else
\renewcommand{\acvHeaderSocialSep}{\quad\textbar\quad}

%-------------------------------------------------------------------------------
%	PERSONAL INFORMATION
%	Comment any of the lines below if they are not required
%-------------------------------------------------------------------------------
% Available options: circle|rectangle,edge/noedge,left/right
% \photo[circle,edge,right]{matteo.png}
\name{Matteo}{Rizzuto}
\position{Visiting Scientist{\enskip\cdotp\enskip}Department of Biology, Memorial University of Newfoundland and Labrador}
% \address{}

\mobile{+1 (709) 986-5944}
\email{matteomrizzuto@gmail.com}
% \dateofbirth{December 12th, 1985}
\homepage{matteorizzuto.github.io}
% \github{matteorizzuto}
% \linkedin{matteorizzuto}
% \gitlab{gitlab-id}
% \stackoverflow{SO-id}{SO-name}
% \twitter{@MatteoRiz}
% \skype{matteolorin}
% \reddit{reddit-id}
% \medium{medium-id}
% \kaggle{kaggle-id}
% \googlescholar{googlescholar-id}{name-to-display}
%% \firstname and \lastname will be used
% \googlescholar{JkHYiIEAAAAJ}{Matteo Rizzuto}
\orcid{0000-0003-3065-9140}
% \researchgate{Matteo Rizzuto}
\extrainfo{\newline Pronouns: he, him, his}

% \quote{``Be the change that you want to see in the world."}


%-------------------------------------------------------------------------------
\begin{document}

% Print the header with above personal information
% Give optional argument to change alignment(C: center, L: left, R: right)
\makecvheader[L]


% Print the footer with 3 arguments(<left>, <center>, <right>)
% Leave any of these blank if they are not needed
\makecvfooter
  {\today}
  {Matteo Rizzuto~~~·~~~Curriculum Vitae}
  {\thepage}


%-------------------------------------------------------------------------------
%	CV/RESUME CONTENT
%	Each section is imported separately, open each file in turn to modify content
%-------------------------------------------------------------------------------
%-------------------------------------------------------------------------------
%	SECTION TITLE
%-------------------------------------------------------------------------------
\cvsection{Highlights}


%-------------------------------------------------------------------------------
%	CONTENT
%-------------------------------------------------------------------------------

\begin{itemize}
  \item Published 13 peer-reviewed papers, of which 4 as first author and one as shared first author, with a combined 136 citations and an h-index of 7 (i10-index of 5).
  \item Winner of \textit{Oecologia}'s 2021 Hanski prize for best student-authored paper in animal ecology (as first author) and of the \textit{Journal of Animal Ecology}'s 2021 Sidnie Manton award for best early-career contribution (as co-author).
\end{itemize}
% %-------------------------------------------------------------------------------
%	SECTION TITLE
%-------------------------------------------------------------------------------
\cvsection{Current position}


%-------------------------------------------------------------------------------
%	CONTENT
%-------------------------------------------------------------------------------
\begin{cventries}
%---------------------------------------------------------
  \cventry
    {\normalsize Laboratory of Community and Quantitative Ecology, Department of Biology, Concordia University} % Organization
    {\normalsize Postdoctoral Fellow} % Job title
    {\normalsize 2024--Present} % Date(s)
    {\normalsize St. John's, NL, Canada} % Location
    {\normalsize
      \begin{cvitems} % Description(s) of tasks/responsibilities
        \item {Developing mathematical models to explore how animal dispersal influences the relationship between biodiversity, community structure, and ecosystem functioning at multiple spatio-temporal scales.}
        \item {}
      \end{cvitems} 
    }
\end{cventries}

%-------------------------------------------------------------------------------
%	SECTION TITLE
%-------------------------------------------------------------------------------
\cvsection{Education}


%-------------------------------------------------------------------------------
%	CONTENT
%-------------------------------------------------------------------------------
\begin{cventries}

%---------------------------------------------------------
  \cventry
    {Memorial University of Newfoundland and Labrador} % Institution
    {Doctor of Philosophy, \href{http://www.mun.ca/biology}{Biology}} % Degree
    {2016--2021} % Date(s)
    {St. John's, NL, Canada} % Location
    {
      \begin{cvitems} % Description(s) bullet points
        \item {Thesis title: From elements to landscapes: the role of terrestrial consumers in ecosystem functioning}
        \item {Advisor: Dr.~Shawn~J.~Leroux}
      \end{cvitems}
    }

  \cventry
    {Imperial College London} % Institution
    {Master of Research (Distinction), \href{https://www.imperial.ac.uk/study/pg/life-sciences/ecology-evolution-conservation-research/}{Ecology, Evolution, and Conservation}} % Degree
    {2013--2014} % Date(s)
    {London, UK} % Location
    {}
  
  \cventry
    {University of Turin} % Institution
    {Master of Science (cum laude), \href{https://goo.gl/rCzbq7}{Evolution of Animal and Human Behaviour}} % Degree
    {2009--2012} % Date(s)
    {Turin, TO, Italy} % Location
    {}
  
  \cventry
    {University of Turin} % Institution
    {Bachelor of Science (Honours), \href{http://biologia.campusnet.unito.it/do/home.pl}{Biology}} % Degree
    {2004--2009} % Date(s)
    {Turin, TO, Italy} % Location
    {}
%---------------------------------------------------------
\end{cventries}

%-------------------------------------------------------------------------------
%	SECTION TITLE
%-------------------------------------------------------------------------------
\cvsection{Research Appointments}


%-------------------------------------------------------------------------------
%	CONTENT
%-------------------------------------------------------------------------------
\begin{cventries}

%---------------------------------------------------------
  \cventry
    {Schmitz Lab, School of the Environment, Yale University} % Organization
    {Visiting Assistant in Research} % Job title
    {Feb.--Apr. 2019} % Date(s)
    {New Haven, USA} % Location
    {
      \begin{cvitems} % Description(s) of tasks/responsibilities
        \item {Developed a meta-ecosystem model capturing the effects of animal-mediated nutrient transport across ecosystem boundaries}
      \end{cvitems}  
    }

%---------------------------------------------------------
  \cventry
    {Pawar Lab, Department of Life Sciences, Imperial College London---Silwood Park} % Organization
    {Research Assistant} % Job title
    {Jan.--Mar. 2016} % Date(s)
    {Ascot, UK} % Location
    {
      \begin{cvitems} % Description(s) of tasks/responsibilities
        \item {Collected and managed data for a meta-analysis of published studies of plant thermal responses}
      \end{cvitems}  
    }

%---------------------------------------------------------
  \cventry
    {Tsaobis Baboon Project, Zoological Society of London} % Organization
    {Research Assistant} % Job title
    {Jun.--Oct. 2015} % Date(s)
    {London, UK} % Location
    {
       \begin{cvitems} % Description(s) of tasks/responsibilities
        \item {Field-based placement to collect behavioral, dietary, and population data from wild \textit{Chacma} baboons in the Namib Desert}
      \end{cvitems}  
    }

%---------------------------------------------------------
  \cventry
    {Roe \& Red Deer in Trentino and Technology Project, Fondazione Edmund Mach} % Organization
    {Research Assistant} % Job title
    {Jun.--Sept. 2013} % Date(s)
    {Trento, Italy} % Location
    {
      \begin{cvitems} % Description(s) of tasks/responsibilities
        \item {Field-based placement to collect data on roe deer movement data, plant phenology, and pellet decay rate in the Eastern Alps}
      \end{cvitems}   
    }

%---------------------------------------------------------
  \cventry
    {Piedmont Wolf Project, Maritime Alps Nature Park} % Organization
    {Graduate Intern} % Job title
    {Jun.--Dec. 2011} % Date(s)
    {Entracque (CN), Italy} % Location
    {
      \begin{cvitems} % Description(s) of tasks/responsibilities
        \item {Designed and led a mixed laboratory and field-based study of wolves-chamois predator-prey interactions in the Western Alps}
      \end{cvitems} 
    }

%---------------------------------------------------------
  \cventry
    {Ethology Lab, University of Turin} % Organization
    {Undergraduate Intern} % Job title
    {2008--2009} % Date(s)
    {Turin, Italy} % Location
    {
       \begin{cvitems} % Description(s) of tasks/responsibilities
        \item {Bio-acoustic study of an individual's contribution to their family group song among \textit{Indri} lemurs}
      \end{cvitems} 
    }

%---------------------------------------------------------
\end{cventries}

%-------------------------------------------------------------------------------
% SECTION TITLE
%-------------------------------------------------------------------------------
\cvsection{Publications}

%-------------------------------------------------------------------------------
% SUBSECTION TITLE
%-------------------------------------------------------------------------------
\descriptionstyle{A {\textdagger} marks equal contributions. When listed as 2\textsuperscript{nd} or 3\textsuperscript{rd} author, I contributed to ideas, design, data collection and analysis, interpretation, and writing. When listed as 4\textsuperscript{th} author or later, I contributed to ideas, data collection, and writing.}

\entrypositionstyle{Peer reviewed} 

\begin{etaremune}
  \renewcommand\labelenumi{\bfseries\theenumi .}
  \item \textcolor{awesome}{Rizzuto, M.}, Leroux, S. J., Schmitz, O. J., Vander Wal, E., Wiersma, Y. F., Heckford, T. R. 2024. \href{https://doi.org/10.1016/j.ecolmodel.2023.110570}{Animal-vectored nutrient flows across resource gradients influence the nature of local and meta-ecosystem functioning}. \emph{Ecological Modelling} \textbf{488}, 110570. 
  \item McLeod, A.M., Leroux, S.J., \textcolor{awesome}{Rizzuto, M.}, Leibold, M.A., Schiesari, L. 2023. Integrating ecosystem and contaminant models to predict the effects of ecosystem fluxes on contaminant dynamics. Accepted, \emph{Ecosphere}, manuscript id: ECS23-0511. \href{https://doi.org/10.1101/2023.07.15.549171}{\emph{bioRxiv preprint}}.
  \item Heckford, T.R., Leroux, S.J., Vander Wal, E., \textcolor{awesome}{Rizzuto, M.}, Balluffi-Fry, J., Richmond, I.C., Wiersma, Y.F. 2022. \href{https://doi.org/10.1002/ece3.9244}{Ecoregion and community structure influences on the foliar elemental niche of balsam fir (\textit{Abies balsamea} (L.) Mill.) and white birch (\textit{Betula papyrifera} Marshall)}. \emph{Ecology and Evolution} \textbf{12}, e9244. 
  \item Little, C.J.\textsuperscript{\textdagger}, \textcolor{awesome}{Rizzuto, M.}\textsuperscript{\textdagger}, Luhring, T.M., Monk, J.D., Nowicki, R.J., Paseka, R.E., Stegen, J.C., Symons, C.C., Taub, F.B., Yan, J.D.L. 2022. \href{https://doi.org/10.1111/oik.08892}{Movement with Meaning: Integrating Information into Meta-Ecology}. \emph{Oikos} \textbf{8}, e08892.\\ \null\hfill\textbf{\textit{Editor's Choice}}
  \item Balluffi-Fry, J., Leroux, S.J., Wiersma, Y.F., Richmond, I.C., Heckford, T.R., \textcolor{awesome}{Rizzuto, M.}, Kennah, J.L., Vander Wal, E. 2022. \href{https://rdcu.be/cAY5a}{Integrating plant stoichiometry and feeding experiments: state-dependent forage choice and its implications on body mass.} \emph{Oecologia} \textbf{198}(3), 579--591.
  \item Richmond, I.C., Balluffi-Fry, J., Vander Wal, E., Leroux, S.J., \textcolor{awesome}{Rizzuto, M.}, Heckford, T.R., Kennah, J.L., Riefesel, G.R., Wiersma, Y.F. 2022. \href{https://academic.oup.com/jmammal/advance-article/doi/10.1093/jmammal/gyab130/6441781?guestAccessKey=8f89e422-7fb9-4ce9-a9dc-ccf46f3dd0cc}{Individual snowshoe hares manage risk differently: integrating stoichiometric distribution models and foraging ecology}. \emph{Journal of Mammalogy} \textbf{103}(1), 196--208.
  \item Heckford, T.R., Leroux, S.J., Vander Wal, E., \textcolor{awesome}{Rizzuto, M.}, Balluffi-Fry, J., Richmond, I.C., Wiersma, Y.F. 2022. \href{https://doi.org/10.1007/s10980-021-01334-3}{Spatially explicit correlates of plant functional traits inform landscape patterns of resource quality}. \emph{Landscape Ecology} \textbf{37}, 59--80.
  \item \textcolor{awesome}{Rizzuto, M.}, Leroux, S.J., Vander Wal, E., Richmond, I.C., Heckford, T.R., Balluffi-Fry, J., Wiersma, Y.F. 2021. \href{https://rdcu.be/cSX31}{Forage stoichiometry predicts the home range size of a small terrestrial herbivore}. \emph{Oecologia} \textbf{197}(2), 327--338.\\ \null\hfill\textbf{\textit{Winner, Hanski Prize 2021}}
  \item Ellis-Soto, D.\textsuperscript{\textdagger}, Ferraro, K.M.\textsuperscript{\textdagger}, \textcolor{awesome}{Rizzuto, M.}, Briggs, E., Monk, J.D., and Schmitz, O.J. 2021. \href{https://doi.org/10.1111/1365-2656.13538}{A methodological roadmap to quantify animal-vectored spatial ecosystem subsidies}. \emph{Journal of Animal Ecology} \textbf{90}(7), 1605--1622.\\ \null\hfill\textbf{\textit{Winner, Sidnie Manton Award 2021}}
  \item Richmond, I.C., Leroux, S.J., Vander Wal, E., Heckford, T.R., \textcolor{awesome}{Rizzuto, M.}, Balluffi-Fry, J., Kennah, J., Wiersma, Y.F. 2021. \href{https://doi.org/10.1093/jpe/rtaa103}{Temporal variation and its drivers in the elemental traits of four boreal plant species}. \emph{Journal of Plant Ecology} \textbf{14}(3), 398--413.
  \item Balluffi-Fry, J., Leroux, S.J., Wiersma, Y.F., Heckford, T.R., \textcolor{awesome}{Rizzuto, M.}, Richmond, I.C., Vander Wal, E. 2020. \href{https://doi.org/10.1002/ece3.6975}{Quantity-quality trade-offs revealed using a multiscale test of herbivore resource selection on elemental landscapes}. \emph{Ecology and Evolution} \textbf{10}(24), 13847--13859.
  \item \textcolor{awesome}{Rizzuto, M.}, Leroux, S.J., Vander Wal, E., Wiersma, Y.F., Heckford, T.R., Balluffi-Fry, J. 2019. \href{https://doi.org/10.1002/ece3.5880}{Patterns and potential drivers of intraspecific variability in the body C, N, and P composition of a terrestrial vertebrate, the snowshoe hare (\textit{Lepus americanus})}. \textit{Ecology and Evolution} \textbf{9}(24), 14453--14464.
  \item \textcolor{awesome}{Rizzuto, M.}, Carbone, C., and Pawar, S. 2018. \href{https://doi.org/10.1038/s41559-017-0386-1}{Foraging constraints reverse the scaling of activity time in carnivores}. \emph{Nature Ecology and Evolution} \textbf{2}(2), 247--253.\\ \null\hfill\textbf{\textit{Cover~Story}}
\end{etaremune}

\entrypositionstyle{In progress}

\begin{itemize} 
  \item \textcolor{awesome}{Rizzuto, M.}, Leroux, S. J., Schmitz, O. J. 2023 Rewiring the carbon cycle: a theoretical framework for animal-driven ecosystem carbon sequestration. In revision, \emph{Journal of Geophysical Research: Biogeosciences}, manuscript id: 2023JG007714. \href{https://doi.org/10.1101/2023.07.14.549071}{\emph{bioRxiv preprint}}.
\end{itemize}

\entrypositionstyle{Outreach}

\begin{itemize}
  \item Wiersma, Y.F., Catto, N., Deal, C., Edinger, E., Evans, R., Geissinger, E., Hearn, C., Sun Lim, K., McCann, N., MacDonald, K., Meyer, A., Prosser, J., Quinn, D., Richmond, I.C., \textcolor{awesome}{Rizzuto, M.}, Roncal, J., Swain, M. 2020. \href{https://nllandscapeecology.com/blog-post-teaching-and-learning-remotely-time-to-read-4-min-45-s/}{The classroom goes virtual---experiences at Memorial University}. \emph{Blog post}.
  \item Cagnacci, F., Rocca, M., Nicoloso, S., Ossi, F., Peters, W., Mancinelli, S., Valent, M., \textcolor{awesome}{Rizzuto, M.}, Hebblewhite, M. 2013. \href{https://en.calameo.com/read/00214567355b1384f96d0}{Il progetto 2C2T}. \emph{Il Cacciatore Trentino}, 93, 4--15.
\end{itemize}


% \begin{refsection}
%   \nocite{Rizzuto2018}
  
%     \printbibliography[
%     heading=none
%     ]
% \end{refsection}

%-------------------------------------------------------------------------------
% SUBSECTION TITLE
%-------------------------------------------------------------------------------
% \cvsubsection{Conference Proceedings}

% \begin{refsection}
%     \nocite{Khan2014Prototyping}
  
%   \printbibliography[
%   heading=none, 
%   sorting=ydnt
%   ]
% \end{refsection}
\newpage
%-------------------------------------------------------------------------------
%	SECTION TITLE
%-------------------------------------------------------------------------------
\cvsection{Presentations}


%-------------------------------------------------------------------------------
%	CONTENT
%-------------------------------------------------------------------------------
\skilltypestyle{Conference Talks}

\begin{itemize}
  \setlength\itemsep{.25em}
  \item \textcolor{awesome}{Rizzuto, M.}, Leroux, S.J., Schmitz, O.J. (2024, Aug. 6--11) \emph{Modeling the zoogeochemical effects of herbivores on ecosystem carbon cycles.} ``For All Ecologists'' Ecological Society of America Annual Meeting, Portland, OR, USA. \textbf{\emph{Invited talk}}  
  \item \textcolor{awesome}{Rizzuto, M.}, Leroux, S.J., Schmitz, O.J. (2023, May 25) \emph{Modeling the zoogeochemical effects of herbivores on ecosystem carbon cycles.} 6\textsuperscript{th} Yale Postdoc Association Annual Symposium, New Haven, CT, USA.
  \item \textcolor{awesome}{Rizzuto, M.}, Leroux, S.J., Schmitz, O.J., Vander Wal, E., Wiersma, Y.F., Heckford, T.R. (2023, May 2--6 ) \emph{Movers and shakers: Animal-vectored nutrient flows across resource gradients influence local and meta-ecosystem functioning.} ``Ecological Models for Tomorrow's Solutions'' The International Society for Ecological Modelling Global Conference 2023, Toronto, ON, Canada.
  \item \textcolor{awesome}{Rizzuto, M.}, Leroux, S.J., Schmitz, O.J., Vander Wal, E., Wiersma, Y.F., Heckford, T.R. (2021, Aug. 2--6) \emph{Going against the flow: non-diffusive organismal movement influences local and meta-ecosystem functioning.} ``Vital Connections in Ecology'' Ecological Society of America Virtual Annual Meeting, Long Beach, CA, USA.
  \item \textcolor{awesome}{Rizzuto, M.}, Leroux, S.J., Vander Wal, E., Wiersma, Y.F., Heckford, T.R., Balluffi-Fry, J. (2018, Jul. 18--21) \emph{Ontogeny and Ecological Stoichiometry of Snowshoe hares (Lepus americanus) in the Boreal Forests of Newfoundland.} Canadian Society for Ecology and Evolution Annual General Meeting, Guelph, ON, Canada.
  \item \textcolor{awesome}{Rizzuto, M.}, Carbone, C., and Pawar, S. (2016, Jul. 7--11) \emph{Bio-mechanical constraints on foraging reverse the scaling of activity rate among carnivores.} Canadian Society for Ecology and Evolution Annual General Meeting, St. John's, NL, Canada.
\end{itemize}

\skilltypestyle{Conference Posters}

\begin{itemize}
  \item \textcolor{awesome}{Rizzuto, M.}, Leroux, S.J., Vander Wal, E., Wiersma, Y.F., Heckford, T.R., Balluffi-Fry, J. (2018, Jul. 21--27) \emph{Beyond Diffusion: Animal-Mediated Nutrient Transport at Different Spatial Scales.} ``Unifying Ecology Across Scales'' Gordon Research Seminar and Conference, Biddeford, ME, USA.
\end{itemize}

\skilltypestyle{Workshops}

\begin{itemize}
  \item \textcolor{awesome}{Rizzuto, M.} (4 Nov. 2022) \emph{Grammar of Graphics: ggplot2.} Part of SPRY: A Learning Community for Quantitative Skill-Sharing\\ Yale University, School of the Environment.
  \item \textcolor{awesome}{Rizzuto, M.} (14 Oct. 2022) \emph{(R)markdown: a brief tour.} Part of SPRY: A Learning Community for Quantitative Skill-Sharing\\ Yale University, School of the Environment.
\end{itemize}

\skilltypestyle{Outreach}

\begin{itemize}
  \item \textcolor{awesome}{Rizzuto, M.} (24 Sept. 2024) \emph{Understanding the evidence. The latest science on trophic rewilding.} ``Are wild animals the unsung heroes of climate action?'' Panel, International Fund for Animal Welfare, New York Climate Week.
\end{itemize}
\begin{itemize}
  \item \textcolor{awesome}{Rizzuto, M.} (26 Apr. 2024) \emph{How animals shape our world: ecosystems, nutrient cycling, and carbon capture.} Wolfram Ecological Research Colloquium.
\end{itemize}

\skilltypestyle{Skype-a-Scientist initiative}

\begin{itemize}
  \item Louisville Zoo, Louisville, KY, USA. (30 Apr. 2025) 
  \item Tri-West Middle School, Lizton, IN, USA. (4 Dec. 2024)
  \item Edward Jones Scientist Expo Week. (16 Apr. 2024)
  \item Hope Township Elementary School, Hope, NJ, USA. (22 Mar. 2023)
\end{itemize}


%-------------------------------------------------------------------------------
%	SECTION TITLE
%-------------------------------------------------------------------------------
\cvsection{Teaching}


%-------------------------------------------------------------------------------
%	CONTENT
%-------------------------------------------------------------------------------
\begin{cventries}

%---------------------------------------------------------
  \cventry
    {Guest Lecturer} % Role
    {Yale University} % Title
    {2022} % Date(s)
    {New Haven, USA} % Location
    {
      \begin{cvsubentries}
       \cvsubentry
         {Industrial Ecology}
         {\footnotesize Graduate course}
         {Fall 2022}
         {
         \begin{cvitems}
          \item Crafted and delivered a lecture introducing ideas and theories from metabolic ecology to urban ecology
         \end{cvitems} 
         }
      \end{cvsubentries}
    }

%---------------------------------------------------------
  \cventry
    {Teaching Assistant and Guest Lecturer} % Role
    {Memorial University of Newfoundland and Labrador} % Title
    {2017--2020} % Date(s)
    {St. John's, Canada} % Location
    {
      \begin{cvsubentries}
       \cvsubentry
         {Models in Biology}
         {\footnotesize Graduate and Undergraduate course}
         {Winter 2020}
         {
         \begin{cvitems}
          \item Designed and delivered lectures on classical models in Ecology, models of species interactions, and meta-ecology models
         \end{cvitems} 
         }
       \cvsubentry
         {Principles of Ecology}
         {\footnotesize Undergraduate course}
         {Fall 2018}
         {
         \begin{cvitems}
          \item Managed the online learning part of the course, provided administrative and academic support to students
         \end{cvitems} 
         }
       \cvsubentry
         {Graduate Core Seminar}
         {\footnotesize Graduate course}
         {Fall 2018}
         {
         \begin{cvitems}
          \item Developed and delivered a seminar on Student-Supervisor Communications
         \end{cvitems} 
         }  
       \cvsubentry
         {Principles of Biology}
         {\footnotesize Undergraduate course}
         {Winter 2017}
         {
         \begin{cvitems}
          \item Assisted during lab-based lectures, marked lab reports, midterms, and invigilated final exams
         \end{cvitems} 
         }
      \end{cvsubentries}
    }

%---------------------------------------------------------
 \cventry
    {Teaching Assistant} % Role
    {Imperial College London} % Title
    {2014--2015} % Date(s)
    {London, UK} % Location
    {
      \begin{cvsubentries}
       \cvsubentry
         {Ecology}
         {\footnotesize Undergraduate course}
         {Spring 2015}
         {
         \begin{cvitems}
          \item Demonstrator for the course's field trip, helped students design, collect data, and run analyses for their final projects
         \end{cvitems} 
         }
       \cvsubentry
         {Behavioral Ecology}
         {\footnotesize Undergraduate course}
         {Winter 2015}
         {
         \begin{cvitems}
          \item Assisted with lab-based lectures, from experiment setup to helping students with lab data analysis
         \end{cvitems} 
         }
       \cvsubentry
         {Introduction to Biological Statistics}
         {\footnotesize Undergraduate course}
         {Fall 2014}
         {
         \begin{cvitems}
          \item Helped students learn base and advanced R language, facilitated Q\&A sessions ahead of final exams
         \end{cvitems} 
         }
       \cvsubentry
        {Statistics}
        {\footnotesize Graduate course}
        {Fall 2014}
        {
          \begin{cvitems}
          \item Helped students in developing and coding statistical analyses in R
         \end{cvitems}
        }  
       \cvsubentry
        {Macroecology and Climate Change}
        {\footnotesize Graduate course}
        {Fall 2014}
        {
          \begin{cvitems}
          \item Assisted students using the species distribution software Maxent
         \end{cvitems}
        }   
      \end{cvsubentries}
    }
%---------------------------------------------------------
\end{cventries}

%-------------------------------------------------------------------------------
%	SECTION TITLE
%-------------------------------------------------------------------------------
\cvsection{Service}\\
\descriptionstyle{Note: 
\emph{Memorial University} stands for Memorial University of Newfoundland and Labrador.}

%-------------------------------------------------------------------------------
%	CONTENT
%-------------------------------------------------------------------------------

\skilltypestyle{Editor}

\begin{cventries}
%---------------------------------------------------------
  \cventry
    {\href{https://esajournals.onlinelibrary.wiley.com/journal/21508925}{Ecosphere}} % Position
    {Subject Matter Editor, General Ecology} % Committee
    {2024--Present} % Location
    {} % Date(s)
    {}
\end{cventries}

\skilltypestyle{Reviewer} %\entrypositionstyle{for \emph{Royal Society Open Sciences}, \emph{Biological Conservation}, \emph{Science of the Total Environment}, \emph{Ecology and Evolution}}

\begin{cvhonors}
%---------------------------------------------------------
  \mycvhonor
    {Journal of Environmental Management} % Position
    {} % Committee
    % {} % Location
    {2024} % Date(s)
    % {}
    
  \mycvhonor
    {Soil Biology and Biogeochemistry} % Position
    {} % Committee
    % {} % Location
    {2024} % Date(s)
    % {}

  \mycvhonor
    {Ecological Modelling} % Position
    {} % Committee
    % {} % Location
    {2024} % Date(s)
    % {}

  \mycvhonor
    {Ecology and Evolution} % Position
    {} % Committee
    % {} % Location
    {2021, 2024} % Date(s)
    % {}

  \mycvhonor
    {Royal Society Open Sciences} % Position
    {} % Committee
    % {} % Location
    {2024} % Date(s)
    % {}  

  \mycvhonor
    {Biological Conservation} % Position
    {} % Committee
    % {} % Location
    {2021, 2022} % Date(s)
    % {}  

  \mycvhonor
    {Science of the Total Environment} % Position
    {} % Committee
    % {} % Location
    {2019} % Date(s)
    % {}
\end{cvhonors}

\skilltypestyle{Conference organization}

\begin{cventries}
%---------------------------------------------------------
  \cventry
    {Yale University} % Position
    {6\textsuperscript{th} and 7\textsuperscript{th} Annual Postdoc Symposium} % Committee
    {2023--2024} % Location
    {New Haven, CT, USA} % Date(s)
    {}
\end{cventries}

\skilltypestyle{Community}

\begin{cventries}

% %---------------------------------------------------------
%   \cventry
%     {Yale Postdoctoral Association} % Affiliation/role
%     {Advisory Board Member} % Organization/group
%     {Present} % Location
%     {Yale University} % Date(s)
%     {
%       \begin{cvitems} % Description(s) of experience/contributions/knowledge
%         \item {Member of the Yale Postdoctoral Association Advisory Board, participating in monthly meetings to steer the Association's activities}
%       \end{cvitems}
%     }

%---------------------------------------------------------
  \cventry
    {Yale University} % Affiliation/role
    {Yale Postdoctoral Association} % Organization/group
    {2022--2024} % Location
    {New Haven, CT, USA} % Date(s)
    {
      \begin{cvitems}
        \item Buddy Program Mentor, 2023--2024
        \item Symposium Committee Co-coordinator, 2023
        \item Symposium Committee Member, 2022 
      \end{cvitems}
    }

      % \begin{cvitems} % Description(s) of experience/contributions/knowledge
      %   \item {Co-coordinated the YPA Symposium Committee in organizing the 6\textsuperscript{th} and 7\textsuperscript{th} Annual Yale Postdoc Symposium}
      %   \item {Member of the Yale Postdoctoral Association Executive Board, participating in monthly meetings to steer the Association's activities}
      % \end{cvitems}
    
%---------------------------------------------------------
  % \cventry
  %   {Yale Postdoctoral Association} % Affiliation/role
  %   {Buddy Program Mentor} % Organization/group
  %   {Present} % Location
  %   {Yale University} % Date(s)
  %   {}

    %   \begin{cvitems} % Description(s) of experience/contributions/knowledge
    %     \item {Offered advice, help, and support to incoming international postdocs}
    %   \end{cvitems}
    
%---------------------------------------------------------
  % \cventry
  %   {Yale Postdoctoral Association} % Affiliation/role
  %   {Symposium Committee Member} % Organization/group
  %   {2022--2023} % Location
  %   {Yale University} % Date(s)
  %   {}

      % \begin{cvitems} % Description(s) of experience/contributions/knowledge
      %   \item {Leader of the Scheduling subcommittee, in charge of developing the Symposium's schedule, outreach and social activities, keynote talk, and panel discussions}
      % \end{cvitems}
    
%---------------------------------------------------------
  \cventry
    {Memorial University} % Affiliation/role
    {Biology Graduate Student Association} % Organization/group
    {2017--2020} % Location
    {St. John's, NL, Canada} % Date(s)
    {
      \begin{cvitems}
          \item Communications officer, 2019--2020
          \item Chairperson, 2018--2019
          \item Seminar Co-coordinator, 2017--2018
      \end{cvitems}
      }

      % \begin{cvitems} % Description(s) of experience/contributions/knowledge
      %   \item {Managed the weekly newsletter, website, event advertisement campaigns, and social media presence of the association}
      % \end{cvitems}
    
%---------------------------------------------------------
  % \cventry
  %   {Biology Graduate Student Association} % Affiliation/role
  %   {Chairperson} % Organization/group
  %   {2018--2019} % Location
  %   {Memorial University} % Date(s)
  %   {

  %   }

      % \begin{cvitems} % Description(s) of experience/contributions/knowledge
      %   \item {Coordinated the activities of the Executive Committee, focusing on fundraising, the welfare and mental health of graduate students, and community outreach}
      %   \item {Liaised with the Head of the Department of Biology to coordinate activities and events}
      % \end{cvitems}
    
%---------------------------------------------------------
  % \cventry
  %   {Biology Graduate Student Association} % Affiliation/role
  %   {Seminar Co-coordinator} % Organization/group
  %   {2017--2018} % Location
  %   {Memorial University} % Date(s)
  %   {}

      % \begin{cvitems} % Description(s) of experience/contributions/knowledge
      %   \item {One-third of the Seminar team, in charge of approaching, scheduling, and managing internal and visiting speakers for the weekly seminar series of the Biology Department}
      % \end{cvitems}
    
%---------------------------------------------------------
  \cventry
    {Associazione O.A.S.I.---Operazione Mato Grosso di Torino} % Affiliation/role
    {Educator} % Organization/group
    {2000--2013} % Location
    {Turin, TO, Italy} % Date(s)
    {}

      % \begin{cvitems} % Description(s) of experience/contributions/knowledge
      %   \item {Designed, scheduled, and run educational activities and community outreach events with and for children, teenagers, and young adults in Turin, Italy, and at the Hospital S{\~a}o Juli{\~a}o in Campo Grande, Mato Grosso do Sul, Brazil}
      % \end{cvitems}
    
%---------------------------------------------------------
\cventry
    {XX Winter Olympic and IX Paralympic Games} % Affiliation/role
    {Team Leader} % Organization/group
    {2005--2006} % Location
    {Turin, TO, Italy} % Date(s)
    {}

      % \begin{cvitems} % Description(s) of experience/contributions/knowledge
      %   \item {Led, scheduled, and supervised the daily activities of a ten-strong team of volunteers within the International Relations and Services Protocol Crew at the Ice Hockey, Curling, and Olympic Ceremonies venues}
      % \end{cvitems}
    
%---------------------------------------------------------
\end{cventries}

\newpage
%-------------------------------------------------------------------------------
%	SECTION TITLE
%-------------------------------------------------------------------------------
\cvsection{Professional Development}\\
\descriptionstyle{Note: 
\emph{Memorial University} stands for Memorial University of Newfoundland and Labrador.} 


%-------------------------------------------------------------------------------
%	CONTENT
%-------------------------------------------------------------------------------

Courses

\begin{cventries}

%---------------------------------------------------------
  \cventry
    {Office for Postdoctoral Affairs} % Affiliation/role
    {Mentorship Training Program for Yale Postdocs} % Organization/group
    {March 2023} % Location
    {Yale University} % Date(s)
    {
      % \begin{cvitems} % Description(s) of experience/contributions/knowledge
      %   \item {}
      % \end{cvitems}
    }

%---------------------------------------------------------
  \cventry
    {Office of Career Strategy} % Affiliation/role
    {Career Development Leaders Program} % Organization/group
    {Jan.--Mar. 2023} % Location
    {Yale University} % Date(s)
    {
      % \begin{cvitems} % Description(s) of experience/contributions/knowledge
      %   \item {}
      % \end{cvitems}
    }

%---------------------------------------------------------
  \cventry
    {MIT Teaching Systems Lab} % Affiliation/role
    {Becoming a More Equitable Educator: Mindsets and Practices} % Organization/group
    {Mar.--Jun. 2020} % Location
    {Attended online at \href{www.edx.org}{edX.org}} % Date(s)
    {
      % \begin{cvitems} % Description(s) of experience/contributions/knowledge
      %   \item {Led the Scheduling subcommittee, in charge of planning and organizing the annual YPA symposium}
      % \end{cvitems}
    }

%---------------------------------------------------------
  \cventry
    {Centre for Innovation in Teaching and Learning} % Affiliation/role
    {Teaching Skills Enhancement Program} % Organization/group
    {Sept. 2019--Aug. 2020} % Location
    {Memorial University} % Date(s)
    {
      % \begin{cvitems} % Description(s) of experience/contributions/knowledge
      %   \item {Managed the Association's weekly newsletter, website, event advertisement campaigns, and social media presence}
      % \end{cvitems}
    }

%---------------------------------------------------------
\end{cventries}

Workshops

\begin{cventries}

%---------------------------------------------------------
  \cventry
    {Yale Postdoc Association and Yale DEI Office} % Affiliation/role
    {Inclusive Leadership Training} % Organization/group
    {30 Aug. 2023} % Location
    {Yale University} % Date(s)
    {
      % \begin{cvitems} % Description(s) of experience/contributions/knowledge
      %   \item {Led the Scheduling subcommittee, in charge of planning and organizing the annual YPA symposium}
      % \end{cvitems}
    }

%---------------------------------------------------------
  \cventry
    {Yale Postdoc Association} % Affiliation/role
    {Bystander Intervention Training} % Organization/group
    {13 Jun. 2023} % Location
    {Yale University} % Date(s)
    {
      % \begin{cvitems} % Description(s) of experience/contributions/knowledge
      %   \item {Led the Scheduling subcommittee, in charge of planning and organizing the annual YPA symposium}
      % \end{cvitems}
    }

%---------------------------------------------------------
  \cventry
    {Being Well at Yale} % Affiliation/role
    {How to Help: Tips from Mental Health First Aid} % Organization/group
    {1 Mar. 2023} % Location
    {Yale University} % Date(s)
    {
      % \begin{cvitems} % Description(s) of experience/contributions/knowledge
      %   \item {Led the Scheduling subcommittee, in charge of planning and organizing the annual YPA symposium}
      % \end{cvitems}
    }

%---------------------------------------------------------
  \cventry
    {Early Career Section} % Affiliation/role
    {Alt-Academic Career Q\&A I} % Organization/group
    {8-9 Feb. 2023} % Location
    {Ecological Society of America} % Date(s)
    {
      % \begin{cvitems} % Description(s) of experience/contributions/knowledge
      %   \item {Led the Scheduling subcommittee, in charge of planning and organizing the annual YPA symposium}
      % \end{cvitems}
    }    

%---------------------------------------------------------
  \cventry
    {Graduate Writing Lab} % Affiliation/role
    {How to Write an Effective Diversity Statement} % Organization/group
    {11 Oct. 2022} % Location
    {Yale University} % Date(s)
    {
      % \begin{cvitems} % Description(s) of experience/contributions/knowledge
      %   \item {Led the Scheduling subcommittee, in charge of planning and organizing the annual YPA symposium}
      % \end{cvitems}
    }

%---------------------------------------------------------
  \cventry
    {Beyond the PhD} % Affiliation/role
    {Building Your Brand Workshop} % Organization/group
    {8 Sept. 2022} % Location
    {Attended online} % Date(s)
    {
      % \begin{cvitems} % Description(s) of experience/contributions/knowledge
      %   \item {Led the Scheduling subcommittee, in charge of planning and organizing the annual YPA symposium}
      % \end{cvitems}
    }


%---------------------------------------------------------
  \cventry
    {Canadian Institute for Ecology and Evolution \& NSERC-CREATE ``Living Data Project''} % Affiliation/role
    {Reproducible Research through Open Science} % Organization/group
    {11 Jun. 2020} % Location
    {Attended online at \href{https://osf.io/p7r5d/}{osf.io}} % Date(s)
    {
      % \begin{cvitems} % Description(s) of experience/contributions/knowledge
      %   \item {Led the Scheduling subcommittee, in charge of planning and organizing the annual YPA symposium}
      % \end{cvitems}
    }

%---------------------------------------------------------
  \cventry
    {American Society for Engineering Education \& NSF INCLUDES Aspire Alliance} % Affiliation/role
    {Teaching Inclusively \& Equitably Online} % Organization/group
    {21 May 2020} % Location
    {Attended online} % Date(s)
    {
      % \begin{cvitems} % Description(s) of experience/contributions/knowledge
      %   \item {Managed the Association's weekly newsletter, website, event advertisement campaigns, and social media presence}
      % \end{cvitems}
    }

%---------------------------------------------------------
  \cventry
    {Centre for Innovation in Teaching and Learning} % Affiliation/role
    {H5P Maker Session} % Organization/group
    {24 Oct. 2019} % Location
    {Memorial University} % Date(s)
    {
      % \begin{cvitems} % Description(s) of experience/contributions/knowledge
      %   \item {Managed the Association's weekly newsletter, website, event advertisement campaigns, and social media presence}
      % \end{cvitems}
    }

%---------------------------------------------------------
  \cventry
    {Centre for Innovation in Teaching and Learning} % Affiliation/role
    {Community of Inquiry Coffee Break: Open Access} % Organization/group
    {23 Oct. 2019} % Location
    {Memorial University} % Date(s)
    {
      % \begin{cvitems} % Description(s) of experience/contributions/knowledge
      %   \item {Managed the Association's weekly newsletter, website, event advertisement campaigns, and social media presence}
      % \end{cvitems}
    }

%---------------------------------------------------------
  \cventry
    {Centre for Innovation in Teaching and Learning} % Affiliation/role
    {Open Access and Scholarly Publishing} % Organization/group
    {22 Oct. 2019} % Location
    {Memorial University} % Date(s)
    {
      % \begin{cvitems} % Description(s) of experience/contributions/knowledge
      %   \item {Managed the Association's weekly newsletter, website, event advertisement campaigns, and social media presence}
      % \end{cvitems}
    }

%---------------------------------------------------------
  \cventry
    {Enhanced Development of the Graduate Experience} % Affiliation/role
    {Four Things to Consider for Graduate Student Teaching} % Organization/group
    {8 Nov. 2018} % Location
    {Memorial University} % Date(s)
    {
      % \begin{cvitems} % Description(s) of experience/contributions/knowledge
      %   \item {Managed the Association's weekly newsletter, website, event advertisement campaigns, and social media presence}
      % \end{cvitems}
    }

%---------------------------------------------------------
  \cventry
    {The Carpentries} % Affiliation/role
    {AARMS CRG Software Carpentry Workshop} % Organization/group
    {27 May 2017} % Location
    {Memorial University} % Date(s)
    {
      % \begin{cvitems} % Description(s) of experience/contributions/knowledge
      %   \item {Managed the Association's weekly newsletter, website, event advertisement campaigns, and social media presence}
      % \end{cvitems}
    }

%---------------------------------------------------------

\end{cventries}

\newpage
%-------------------------------------------------------------------------------
%	SECTION TITLE
%-------------------------------------------------------------------------------
\cvsection{Honors and Awards}

%-------------------------------------------------------------------------------
%	CONTENT
%-------------------------------------------------------------------------------
\begin{cvhonors}

%---------------------------------------------------------
  \mycvhonor
    {Hanski Prize,} % Award
    {Oecologia} % Event
    % {} % Location
    {2021} % Date(s)
    % {} % Description
    % Awarded by the journal's Editorial Board to the best student-authored paper in animal ecology
%---------------------------------------------------------
  \mycvhonor
    {Mitacs Research Training Award,} % Award
    {Mitacs, St. John's, Canada} % Event
    % {St. John's, NL, Canada} % Location
    {2020} % Date(s)
    % {} % Description
    % Awarded to graduate students to conduct research during the COVID-19 pandemic
%---------------------------------------------------------
  % \mycvhonor
  %   {First Place--H5P Maker Session,} % Award
  %   {Memorial University, St.~John's, Canada} % Event
  %   % {St. John's, NL, Canada} % Location
  %   {2019} % Date(s)
  %   % {} % Description
  %   % Awarded by the Centre for Innovation in Teaching \& Learning to the creator of the best HTML5 interactive learning object
%---------------------------------------------------------
  \mycvhonor
    {Mitacs Globalink Research Award,} % Award
    {Mitacs, St. John's, Canada} % Event
    % {St. John's, NL, Canada} % Location
    {2019} % Date(s)
    % {} % Description
    % Awarded to senior graduate students to conduct research  abroad
%---------------------------------------------------------
  \mycvhonor
    {Dean's Doctoral Award,} % Award
    {Memorial University of Newfoundland and Labrador, St. John's, Canada} % Event
    % {St. John's, NL, Canada} % Location
    {2016--2020} % Date(s)
    % {} % Description
    % Awarded to Ph.D. students based on academic excellence and research potential
%---------------------------------------------------------
  \mycvhonor
    {Graduated with Distinction,} % Award
    {Imperial College London, London, UK} % Event
    % {London, United Kingdom} % Location
    {2014} % Date(s)
    % {} % Description
    % Awarded to the top 5\% of all Master's students in a cohort
%---------------------------------------------------------
  \mycvhonor
    {Erasmus-LLP Scholarship,} % Award
    {University of Turin, Turin, Italy} % Event
    % {Turin, Italy} % Location
    {2010} % Date(s)
    % {} % Description
    % Awarded to students to take part in the Erasmus exchange program of the European Union
%--------------------------------------------------------- 
\end{cvhonors}
% %-------------------------------------------------------------------------------
%	SECTION TITLE
%-------------------------------------------------------------------------------
\cvsection{Extracurricular Activity}\\
\descriptionstyle{Note: 
\emph{Memorial University} stands for Memorial University of Newfoundland and Labrador.}

%-------------------------------------------------------------------------------
%	CONTENT
%-------------------------------------------------------------------------------
\begin{cventries}

%---------------------------------------------------------
  \cventry
    {Yale Postdoctoral Association} % Affiliation/role
    {Symposium Committee Co-coordinator} % Organization/group
    {Present} % Location
    {Yale University} % Date(s)
    {
      \begin{cvitems} % Description(s) of experience/contributions/knowledge
        \item {Co-coordinated the activities of the YPA Symposium Committee and organized the Annual Yale Postdoc Symposium}
        \item {Member of the Yale Postdoctoral Association Executive Board, participating in monthly meetings to steer the Association's activities}
      \end{cvitems}
    }
%---------------------------------------------------------
  \cventry
    {Yale Postdoctoral Association} % Affiliation/role
    {Symposium Committee Member} % Organization/group
    {2022--2023} % Location
    {Yale University} % Date(s)
    {
      \begin{cvitems} % Description(s) of experience/contributions/knowledge
        \item {Leader of the Scheduling subcommittee}
        \item {Developed the Symposium's schedule, planned and organized outreach and social activities, coordinated the search for the keynote speaker and panelists}
      \end{cvitems}
    }

%---------------------------------------------------------
  \cventry
    {Biology Graduate Student Association} % Affiliation/role
    {Communications Officer} % Organization/group
    {2019--2020} % Location
    {Memorial University} % Date(s)
    {
      \begin{cvitems} % Description(s) of experience/contributions/knowledge
        \item {Managed the Association's weekly newsletter, website, event advertisement campaigns, and social media presence}
      \end{cvitems}
    }

%---------------------------------------------------------
  \cventry
    {Biology Graduate Student Association} % Affiliation/role
    {Chairperson} % Organization/group
    {2018--2019} % Location
    {Memorial University} % Date(s)
    {
      \begin{cvitems} % Description(s) of experience/contributions/knowledge
        \item {Coordinated the activities of the Executive Committee, focusing on fundraising, the welfare and mental health of graduate students, and community outreach}
        \item {Liaised with the Head of the Department of Biology to coordinate activities and events}
      \end{cvitems}
    }

%---------------------------------------------------------
  \cventry
    {Biology Graduate Student Association} % Affiliation/role
    {Seminar Co-coordinator} % Organization/group
    {2017--2018} % Location
    {Memorial University} % Date(s)
    {
      \begin{cvitems} % Description(s) of experience/contributions/knowledge
        \item {One-third of the Seminar team, in charge of approaching, scheduling, and managing internal and visiting speakers for the weekly seminar series of the Biology Department}
      \end{cvitems}
    }

%---------------------------------------------------------
  \cventry
    {Associazione O.A.S.I.---Operazione Mato Grosso di Torino} % Affiliation/role
    {Educator} % Organization/group
    {2000--2013} % Location
    {Turin, Italy} % Date(s)
    {
      \begin{cvitems} % Description(s) of experience/contributions/knowledge
        \item {Designed, scheduled, and run educational activities and community outreach events with and for children, teenagers, and young adults in Turin, Italy, and at the Hospital S{\~a}o Juli{\~a}o in Campo Grande, Mato Grosso do Sul, Brazil}
      \end{cvitems}
    }

%---------------------------------------------------------
\cventry
    {XX Winter Olympic and IX Paralympic Games} % Affiliation/role
    {Team Leader} % Organization/group
    {2005--2006} % Location
    {Turin, Italy} % Date(s)
    {
      \begin{cvitems} % Description(s) of experience/contributions/knowledge
        \item {Led, scheduled, and supervised the daily activities of a ten-strong team of volunteers within the International Relations and Services Protocol Crew at the Ice Hockey, Curling, and Ceremonies Olympic venues}
      \end{cvitems}
    }

%---------------------------------------------------------
\end{cventries}

%-------------------------------------------------------------------------------
%	SECTION TITLE
%-------------------------------------------------------------------------------
\cvsection{Professional Affiliations}


%-------------------------------------------------------------------------------
%	CONTENT
%-------------------------------------------------------------------------------
\begin{cventries}

%---------------------------------------------------------
  \cventry
    {} % Position
    {Ecological Society of America} % Committee
    {2021--2022} % Location
    {} % Date(s)
    {
      %
    }

%---------------------------------------------------------
  \cventry
    {} % Position
    {Canadian Society for Ecology and Evolution} % Committee
    {2016--2021} % Location
    {} % Date(s)
    {
      %
    }

%---------------------------------------------------------
\end{cventries}

% %-------------------------------------------------------------------------------
%	SECTION TITLE
%-------------------------------------------------------------------------------
\cvsection{Conferences}


%-------------------------------------------------------------------------------
%	CONTENT
%-------------------------------------------------------------------------------

Organized

\begin{cventries}
%---------------------------------------------------------
  \cventry
    {6\textsuperscript{th} Annual Symposium} % Position
    {Yale Postdoc Association} % Committee
    {25 May 2023} % Location
    {New Haven, CT, USA} % Date(s)
    {
      %
    }
\end{cventries}

Attended

\begin{cventries}

%---------------------------------------------------------
  \cventry
    {The International Society for Ecological Modelling Global Conference 2023} % Position
    {Ecological Models for Tomorrow's Solutions} % Committee
    {2--6 May 2023} % Location
    {Toronto, ON, Canada} % Date(s)
    {
      %
    }

%---------------------------------------------------------
  \cventry
    {Ecological Society of America, Virtual Annual Meeting} % Position
    {Vital Connections in Ecology} % Committee
    {2--6 Aug. 2021} % Location
    {Long Beach, CA, USA} % Date(s)
    {
      %
    }

%---------------------------------------------------------
  \cventry
    {Gordon Research Seminar and Conference} % Position
    {Unifying Ecology Across Scales} % Committee
    {21--27 Jul. 2018} % Location
    {Biddeford, ME, USA} % Date(s)
    {
      %
    }

%---------------------------------------------------------
  \cventry
    {Canadian Society for Ecology and Evolution} % Position
    {Annual General Meeting} % Committee
    {18--21 Jul. 2018} % Location
    {Guelph, ON, Canada} % Date(s)
    {
      %
    }

%---------------------------------------------------------
  \cventry
    {Canadian Society for Ecology and Evolution} % Position
    {Annual General Meeting} % Committee
    {7--11 Jul. 2016} % Location
    {St. John's, NL, Canada} % Date(s)
    {
      %
    }

%---------------------------------------------------------
  \cventry
    {Zoological Society of London} % Position
    {From Energetics to Macro Ecology: Carnivore Responses to Environmental Change} % Committee
    {14--15 Nov. 2013} % Location
    {London, UK} % Date(s)
    {
      %
    }

%---------------------------------------------------------
  \cventry
    {Euroscience} % Position
    {ESOF---Euroscience Open Forum} % Committee
    {2--7 Jul. 2010} % Location
    {Turin, Italy} % Date(s)
    {
      %
    }

%---------------------------------------------------------
  \cventry
    {Italian Primatological Society} % Position
    {XIX Congress} % Committee
    {1--3 Apr. 2009} % Location
    {Asti, Italy} % Date(s)
    {
      %
    }

%---------------------------------------------------------
\end{cventries}

% %-------------------------------------------------------------------------------
%	SECTION TITLE
%-------------------------------------------------------------------------------
\cvsection{Skills}


%-------------------------------------------------------------------------------
%	CONTENT
%-------------------------------------------------------------------------------
\begin{cvskills}

%---------------------------------------------------------
  \cvskill
    {DevOps} % Category
    {AWS, Docker, Kubernetes, Rancher, Vagrant, Packer, Terraform, Jenkins, CircleCI} % Skills

%---------------------------------------------------------
  \cvskill
    {Back-end} % Category
    {Koa, Express, Django, REST API} % Skills

%---------------------------------------------------------
  \cvskill
    {Front-end} % Category
    {Hugo, Redux, React, HTML5, LESS, SASS} % Skills

%---------------------------------------------------------
  \cvskill
    {Programming} % Category
    {Node.js, Python, JAVA, OCaml, LaTeX} % Skills

%---------------------------------------------------------
  \cvskill
    {Languages} % Category
    {Korean, English, Japanese} % Skills

%---------------------------------------------------------
\end{cvskills}

% %-------------------------------------------------------------------------------
%	SECTION TITLE
%-------------------------------------------------------------------------------
\cvsection{Certificates}


%-------------------------------------------------------------------------------
%	CONTENT
%-------------------------------------------------------------------------------
\begin{cventries}

%---------------------------------------------------------
  \cventry
    {Canadian Red Cross} % Position
    {Wilderness and Remote First Aid} % Committee
    {2016--2019} % Location
    {St. John's, NL, Canada} % Date(s)
    {
      %
    }

%---------------------------------------------------------
  \cventry
    {Memorial University of Newfoundland and Labrador} % Position
    {WHMIS and Lab Safety} % Committee
    {2017} % Location
    {St. John's, NL, Canada} % Date(s)
    {
      %
    }

%---------------------------------------------------------
  \cventry
    {Marlin Training Ltd.} % Position
    {Basic Outdoor First Aid} % Committee
    {2014} % Location
    {London, UK} % Date(s)
    {
      %
    }

%---------------------------------------------------------
  \cventry
    {British Council} % Position
    {International English Language Testing System} % Committee
    {2013} % Location
    {Turin, Italy} % Date(s)
    {
      Grade: 8.5
    }

%---------------------------------------------------------
  \cventry
    {University of Turin} % Position
    {International Computer Driving Licence} % Committee
    {2007} % Location
    {Turin, Italy} % Date(s)
    {
      %
    }

%---------------------------------------------------------
\end{cventries}

% %-------------------------------------------------------------------------------
%	SECTION TITLE
%-------------------------------------------------------------------------------
\cvsection{Interests}


%-------------------------------------------------------------------------------
%	CONTENT
%-------------------------------------------------------------------------------

%---------------------------------------------------------
\begin{cvskills}
  \skillsetstyle{Hiking, Photography, Yoga, Watercolor painting}
\end{cvskills}
% \newpage
%-------------------------------------------------------------------------------
%	SECTION TITLE
%-------------------------------------------------------------------------------
\cvsection{References}


%-------------------------------------------------------------------------------
%	CONTENT
%-------------------------------------------------------------------------------
\cvdoublecolumn
{\cvreference{Dr. Oswald J. Schmitz}
  {School of the Environment}
  {Yale University}
  {195 Prospect Street\\
  New Haven, CT\\
  USA 06511}
  {oswald.schmitz@yale.edu}
  {+1 (203) 436-5276}
}
{\cvreference{Dr. Shawn J. Leroux}
  {Department of Biology}
  {Memorial University of Newfoundland and Labrador}
  {45 Arctic Avenue\\
  St. John's, NL\\
  Canada A1C 5S7}
  {sleroux@mun.ca}
  {+1 (709) 864-3042}
}
{\cvreference{Dr. Chelsea J. Little}
  {School of Environmental Science \& School of Resource and Environmental Management}
  {Simon Fraser University}
  {8888 University Drive\\
  Burnaby, BC\\
  Canada V5A 1S6}
  {chelsea\_little@sfu.ca}
  {+1 (778) 782-5391}
}
{\cvreference{Dr.~Samraat Pawar}
  {Department of Life Sciences}
  {Imperial College London at Silwood Park}
  {Buckhurst Road, Ascot, Berkshire\\
  UK SL5 7PY}
  {s.pawar@imperial.ac.uk}
  {+44 (0) 2075942213}
}

% {\cvreference{Dr. Yolanda F. Wiersma}
%   {Department of Biology}
%   {Memorial University of Newfoundland}
%   {St. John's, NL, Canada}
%   {ywiersma@mun.ca}
% }


%-------------------------------------------------------------------------------
\end{document}
